%%%%%%%%%%%%%%%%%%%%%%%%%%%%%%%%%%%%%%%%%%%%%%%%%%%%%%%%%%%%%%%%%%%%%%%%%%%%%%%%%%%%%%%%%%%%%%%%%%%%%%%%%%%%%%
%%	Licencia:		Este documento se distribuye bajo licencia (CC BY 4.0) Usted puede encontrar un resumen %%
%%					de la licencia en http://creativecommons.org/licenses/by/4.0/deed.es					%%
%%																											%%
%%	-----------------------------------------------------------------------------------------------------	%%
%%																											%%
%%	Plantilla:		exi-examen																				%%
%%	Descripción:	exi-examen.tex es una plantilla para realizar exámenes escritos en formato LaTeX		%%
%%	Autor:			Ing. Alfonso Ramos Michel																%%
%%	Fecha:			9 de diciembre de 2013																	%%
%%	Revisión:		UNO-A																					%%
%%	Paquetes:		inputenc																				%%
%%					babel																					%%
%%					graphicx																				%%
%%					amsmath																					%%
%%					geometry 																				%%
%%					fancyhdr																				%%
%%	Comandos:		\ExiEscuela		Para definir la institución educativa donde se realiza el examen.		%%
%%					\ExiCarrera		Para definir la carrera que se realiza.									%%
%%					\ExiMateria		Para definir la materia que se cursa.									%%
%%					\ExiParcial		Para definir el parcial del cual se realiza el examen.					%%
%%					\informacion	Muesta un recuadro donde poder dar instrucciones al alumno.				%%
%%					\datos			Muestra una línea para el nombre del alumno y la fecha de realización.	%%
%%					\pregunta		Permite introducir (dentro del entorno enummerate) una pregunta o el 	%%
%%									planteamiento de un problema.											%%
%%%%%%%%%%%%%%%%%%%%%%%%%%%%%%%%%%%%%%%%%%%%%%%%%%%%%%%%%%%%%%%%%%%%%%%%%%%%%%%%%%%%%%%%%%%%%%%%%%%%%%%%%%%%%%




\documentclass[letterpaper,11pt]{article}

\usepackage[utf8]{inputenc}
\usepackage[spanish,mexico]{babel}
\usepackage{graphicx}
\usepackage{amsmath}
\usepackage{amsthm}
\usepackage{amsfonts}
%\usepackege{amssymb}
\usepackage[hmargin=1in, vmargin=1in]{geometry}
\usepackage{fancyhdr}
\pagestyle{fancy}
\usepackage{tasks}
\lhead{\ExiCarrera}
\chead{\ExiMateria}
\rhead{\ExiParcial}
\cfoot{\ExiEscuela}
\renewcommand{\headrulewidth}{0.4pt}
\renewcommand{\footrulewidth}{0.4pt}

\providecommand{\abs}[1]{\lvert#1\rvert}
\providecommand{\norm}[1]{\lVert#1\rVert}

%=================================================================================
%	Definición de comandos
%=================================================================================
\newcommand{\informacion}[1]{
\begin{center}
\fbox{\fbox{\parbox{\textwidth}{{\footnotesize#1}}}}
\end{center}
\vspace{5mm}}
\newcommand{\datos}{\makebox[0.7\textwidth]{Nombre:~\hrulefill} Fecha:~\hrulefill}
\newcommand{\pregunta}[2]{\item{#2}~{(#1 puntos)}\\ \vspace{5mm}						       
			{\bf Solución}}														  
%	Los siguientes comandos hay que definirlos desde aquí.								    %
\newcommand{\ExiCarrera}{Matemáticas para las Ciencias II.}											  %%%
\newcommand{\ExiMateria}{\textbf{Tarea I}}														%%%%%%%%%%%%%%%%%%%%%%%%%%%%%
\newcommand{\ExiParcial}{Entrega: 7 de marzo de 2020}														  %%%
\newcommand{\ExiEscuela}{\textbf{Facultad de Ciencias, UNAM}}	                		    %

%=================================================================================
%
%=================================================================================
\begin{document}



%%%%%%%%%%%%%%%%%%%%%            CARÁTULA            %%%%%%%%%%%%%%%%%%%%%%%%%
\setlength{\unitlength}{1cm}
\thispagestyle{empty}
\begin{picture}(18,4)
\put(-0.5,1.2){\includegraphics[scale=.25]{unam1.png}}
\put(13.5,1){\includegraphics[scale=.35]{fciencias1.png}}
\end{picture}

\begin{center}
\vspace{-134pt}
\textbf{\large Matemáticas para las Ciencias I}\\[0.2cm]
\textbf{ Semestre 2020-2}\\[0.2cm]
Prof. Pedro Porras Flores\\[0.2cm]
Ayud. Irving Hernández Rosa \\ [0.2cm]
\textbf{Tarea I}
\end{center}
\vspace{-10pt}
\rule{17cm}{0.3mm}
\begin{flushright}
\vspace{-3pt}
\end{flushright}

%%%%%%%%%%%%%%%%%%%%%%%%%%%%%%%%%%%%%%%%%%%%%%%%%%%%%%%%%

\noindent Realice los siguientes ejercicios, escribiendo el procedimiento claramente. Y recuerden que la tarea se entrega en equipos de a lo más tres integrantes. 


\begin{enumerate}
 
 % -----------------------------------------------------
% Problema uno
% -----------------------------------------------------

\item Escribe el vector cero en $M_{3x4}(\mathbb{R})$

% -----------------------------------------------------
% Problema dos
% -----------------------------------------------------

 \item Sea $V$ el conjunto de todas las funciones diferenciables definidas en $\mathbb{R}$. Muestre que $V$ es un espacio vectorial con las operaciones usuales de suma y multiplicación por un escalar para funciones. 

% -----------------------------------------------------
% Problema tres
% -----------------------------------------------------
 
 \item Prueba que el conjunto de las funciones pares en $\mathbb{R}$ es un espacio vectorial con suma y multiplicación por escalar usuales para funciones. Recuerde que una función es par si $\forall x \in Dom(f)$ entonces $f(-x) = f(x)$ 

% -----------------------------------------------------
% Problema cuatro 
% -----------------------------------------------------

\item Sea $V$ el conjunto de pares ordenados de números reales. Si $(a_{1},a_{2})$ y $(b_{1},b_{2})$ son elementos de $V$ y $\alpha \in \mathbb{R}$, definamos la suma y multiplicación escalar de la siguiente manera:
\begin{enumerate}
\item[(i)] $(a_{1},a_{2}) + (b_{1},b_{2}) = (a_{1} + b_{1}, a_{2}b_{2})$ 
\item[(ii)] $\alpha(a_{1},a_{2}) = (\alpha a_{1},a_{2})$.\\
\end{enumerate}
¿Es $V$ un espacio vectorial sobre $\mathbb{R}$ con estas operaciones?

% -----------------------------------------------------
% Problema cinco
% -----------------------------------------------------

\item Determinar cuales de los siguientes conjuntos son subespacios de $\mathbb{R}^{3}$ bajo las operaciones de suma y multiplicación por un escalar usual.\\

\begin{tasks}(1)
\task $W_{1} = \lbrace (a_{1},a_{2},a_{3}) \in \mathbb{R}^{3} \big\vert  a_{1}=3a_{2}$ y $a_{3}=-a_{2} \rbrace$
\task $W_{2} = \lbrace (a_{1},a_{2},a_{3}) \in \mathbb{R}^{3} \big\vert  a_{1} = a_{3} + 2 \rbrace$
\task $W_{3} = \lbrace (a_{1},a_{2},a_{3}) \in \mathbb{R}^{3} \big\vert  2a_{1} - 7a_{2} + a_{3} = 0 \rbrace$
\task $W_{4} = \lbrace (a_{1},a_{2},a_{3}) \in \mathbb{R}^{3} \big\vert  a_{1} - 4a_{2} - a_{3} = 0 \rbrace$
\end{tasks}
%\newpage
% -----------------------------------------------------
% Problema seis
% -----------------------------------------------------
\item En cada caso diga si los vectores son generados por el conjunto $S$

\begin{tasks}(1)
\task $(2,-1,1), S =  \lbrace (1,0,2),(-1,1,1) \rbrace$
\task $(2,-1,1,3), S =  \lbrace (1,0,1,-1),(0,1,1,1) \rbrace$
\task $2x^3 - x^2 + x + 3, S =  \lbrace x^3 + x^2 + x +1, x^2 + x +1, x +1 \rbrace$
\task $ \begin{pmatrix} 1 & 2 \\ -3 & 4 \end{pmatrix},  S =  \left \lbrace \begin{pmatrix} 1 & 0 \\ -1 & 0 \end{pmatrix} , \begin{pmatrix} 0 & 1 \\ 0 &1 \end{pmatrix} , \begin{pmatrix} 1 & 1 \\ 0 &0 \end{pmatrix} \right \rbrace$
\end{tasks}

% -----------------------------------------------------
% Problema siete
% -----------------------------------------------------

\item Determina cuando los siguientes conjuntos son linealmente dependientes o linealmente independientes.

\begin{tasks}(1)
\task $\left \lbrace \begin{pmatrix} 1 & -3 \\ -2 & 4 \end{pmatrix} , \begin{pmatrix} -2 & 6 \\ 4 & -8 \end{pmatrix} \right \rbrace \in M_{2x2}(\mathbb{R})$
\task $\left \lbrace \begin{pmatrix} 1 & -2 \\ -1 & 4 \end{pmatrix} ,\begin{pmatrix} -1 & 1 \\ 2 & -4 \end{pmatrix}  \right \rbrace \in M_{2x2}(\mathbb{R})$
\task $\lbrace x^{3} + 2x^{2}, -x^{2} + 3x + 1, x^{3} - x^{2} + 2x -1 \rbrace \in P_{3}(\mathbb{R})$ \task $\lbrace (1,-1,2), (1,-2,1), (1,1,4) \rbrace \in \mathbb{R}^{3}$
\task $\lbrace (1,-1,2), (2,0,1),(-1,2,-1)\rbrace \in \mathbb{R}^{3}$
\end{tasks}
Recuerde que $P_{n}(\mathbb{R}) = \{ a_0 +  a_1x +  a_2x^2 + \cdots +  a_n x^n \big\vert a_k \in \mathbb{R} \,  \forall k = 0,1,2,\dots n\}$

% -----------------------------------------------------
% Problema ocho
% -----------------------------------------------------
\item ¿Cuáles de los siguientes conjuntos son bases para $\mathbb{R}^{3}$?
\begin{tasks}(1)
\task  $\lbrace (1,0,-1),(2,5,1),(0,-4,3) \rbrace$
\task $\lbrace (2,-4,1),(0,3,-1),(6,0,-1) \rbrace$
\task $\lbrace (1,2,-1),(1,0,2),(2,1,1) \rbrace$
\end{tasks}

% -----------------------------------------------------
% Problema nueve
% -----------------------------------------------------

\item Diga si los siguientes $x^{3} - 2x^{2} +1$,  $4x^{2} - x + 3$ y $3x - 2$ generan a $P_{3}(\mathbb{R})$

% -----------------------------------------------------
% Problema diez
% -----------------------------------------------------
\item Prueba que las siguientes tranformaciones $T$ son lineales y encuentra el núcleo $Nu(T)$ y la imagen $Im(T)$

\begin{tasks}(1)
\task $\lbrace T: \mathbb{R}^{3} \longrightarrow \mathbb{R}^{2}$ definida por $T(a_{1},a_{2},a_{3}) = (a_{1} - a_{2}, 2a_{3})$
\task $\lbrace T: \mathbb{R}^{2} \longrightarrow \mathbb{R}^{3}$ definida por $(a_{1},a_{2}) = (a_{1} + a_{2}, 0, 2a_{1} - a_{2})$
\task $\lbrace T: M_{2x3}(\mathbb{R}) \longrightarrow M_{2x2}(\mathbb{R})$ definido por\\
\\
$T \begin{pmatrix} a_{11} & a_{12} & a_{13} \\ a_{21} & a_{22} & a_{23} \end{pmatrix} = \begin{pmatrix} 2a_{11} - a_{12} & a_{13} + 2a_{12} \\ 0 & 0 \end{pmatrix}$
\\
\task $T: P_{2}(\mathbb{R}) \longrightarrow P_{3}(\mathbb{R})$ definida por $T(f(x)) = xf(x) + f'(x)$.
\end{tasks}
% -----------------------------------------------------
% Problema once
% -----------------------------------------------------
 
\item Sean $\beta$ y $\gamma$ las bases estándar para $\mathbb{R}^{n}$ y $\mathbb{R}^{m}$ respectivamente. Para cada transformación lineal $T: \mathbb{R}^{n} \longrightarrow \mathbb{R}^{m}$ encontrar su representación matricial.
\begin{tasks}(1)
\task $T: \mathbb{R}^{2} \longrightarrow \mathbb{R}^{3}$ definido por $T(a_{1},a_{2}) = (2a_{1} - a_{2}, 3a_{1} + 4a_{2},a_{1})$
\task $T: \mathbb{R}^{3} \longrightarrow \mathbb{R}^{2}$ definido por $T(a_{1},a_{2},a_{3}) = (2a_{1} + 3a_{2} - a_{3}, a_{1} + a_{3})$
\task $T: \mathbb{R}^{3} \longrightarrow \mathbb{R}$ definido por $T(a_{1},a_{2},a_{3}) = 2a_{1} + a_{2} - 3a_{3}$
\task $T: \mathbb{R}^{3} \longrightarrow \mathbb{R}^{3}$ definido por $T(a_{1},a_{2},a_{3}) = (2a_{2} + a_{3}, -a_{1} + 4a_{2} + 5a_{3},a_{1} + a_{3})$
\end{tasks}

\newpage
% -----------------------------------------------------
% Problema doce
% -----------------------------------------------------
\item Para cada uno de los siguientes pares de bases $\beta$ y $\beta'$ para $\mathbb{R}^{2}$, encuentra la matriz de cambio de coordenadas que cambia las coordenadas de $\beta'$ en las de $\beta$.

\begin{tasks}(1)
\task $\beta = \lbrace \hat{e}_{1},\hat{e}_{2} \rbrace$ y $\beta' = \lbrace (a_{1},a_{2}),(b_{1},b_{2}) \rbrace$
\task $\beta = \lbrace (-1,3),(2,-1) \rbrace$ y $\beta' = \lbrace (0,10),(5,0) \rbrace$
\task $\beta = \lbrace (2,5),(-1,-3) \rbrace$ y $\beta' = \lbrace e_{1},e_{2} \rbrace$
\task $\beta = \lbrace (-4,3),(2,-1) \rbrace$ y $\beta' = \lbrace (2,1),(-4,1) \rbrace$
\end{tasks}

% -----------------------------------------------------
% Problema trece
% -----------------------------------------------------
\item Encontrar la matriz inversa por el método de Gauss-Jordan de las siguientes matrices 

\begin{tasks}(5)
\task $\begin{pmatrix} 1 & 2 \\ 1 & 1 \end{pmatrix}$     
\task $\begin{pmatrix} 1 & 2 \\ 2 & 4 \end{pmatrix}$ 
\task $\begin{pmatrix} 1 & 2 & 1 \\ 1 & 3 & 4 \\ 2 & 3 & -1 \end{pmatrix}$
\task $\begin{pmatrix} 0 & -2 & 4 \\ 1 & 1 & -1 \\ 2 & 4 & -5 \end{pmatrix}$
\end{tasks}


% -----------------------------------------------------
% Problema catorce
% -----------------------------------------------------

\item Calcular el determinante de las siguientes matrices 
\begin{tasks}(4)
\task $\begin{pmatrix} 6 & -3 \\ 2 & 4 \end{pmatrix}$
\task $\begin{pmatrix} -5 & 2 \\ 6 & 1 \end{pmatrix}$ 
\task $\begin{pmatrix} 0 & 1 & 2 \\ -1 & 0 & -3 \\ 2 & 3 & 0 \end{pmatrix}$ 
\task $\begin{pmatrix} 0 & 2 & 1 & 3 \\ 1 & 0 & -2 & 2 \\ 3 & -1 & 0 & 1 \\ -1 & 1 & 2 & 0 \end{pmatrix}$ 
\end{tasks}
% ---------------------------------------------------
% Problema quince
% ----------------------------------------------------
\item Para cada par de vectores $u$ y $v$ en $\mathbb{R}^2$, calcula el área del paralelogramo determinado por $u$ y $v$.

\begin{tasks}(3)
\task $\vec{u} = (3,-2)$ y $\vec{v} = (2,5)$
\task $\vec{u}  = (1,3)$ y $\vec{v}  = (-3,1)$ 
\task $\vec{u}  = (4,-1)$ y $\vec{v}  = (-6,-2)$
\end{tasks}
% ---------------------------------------------------
% Problema diesciseis 
% ----------------------------------------------------
\item Para cada una de las siguientes matrices $A \in M_{nxn}(\mathbb{R})$ determine los valores propios de $A$ y para cada valor propio $\lambda$ de $A$, encontrar el conjunto de vectores propios correspondientes a $A$

\begin{tasks}(3)
\task $A = \begin{pmatrix} 1 & 2 \\ 3 & 2 \end{pmatrix}$
%\task $A = \begin{pmatrix} 2 & 2 \\ 2 & -1 \end{pmatrix}$
\task $A = \begin{pmatrix} 1 & 0 & 2 \\ -1 & 1 & 1 \\ 2 & 0 & 1 \end{pmatrix}$
\task $A = \begin{pmatrix} 0 & -2 & -3 \\ -1 & 1 & -1 \\ 2 & 2 & 5 \end{pmatrix}$
\end{tasks}

% ---------------------------------------------------
% Problema siete
% ----------------------------------------------------

\item Encuentre los ejes principales de la siguiente superficie $x^2 + 4y^2 + 5z^2 + 8xz - 36 = 0$ y con ello construya una matriz de rotación de tal manera que los ejes principales de la superficie coincidan con los con los vectores canónicos de $\mathbb{R}^3$. ¿Qué superficie es?   

\item Aplique el proceso de Gram-Schmidt para construir una base ortonormal en cada caso

\begin{tasks}(2)
\task  $\lbrace (1,1), (1,2) \rbrace$
\task $\lbrace (3,-3), (3,1) \rbrace$
\task $\lbrace (1,-1,-1), (0,3,3), (3,2,4) \rbrace$
\task $\lbrace (1,1,1), (1,1,0), (1,0,0) \rbrace$
\end{tasks}

\item Encuentre la distancia de punto $(2,1,-1)$ al plano $x -2y + 2z + 5 = 0$

\item Encuentre la ecuación del plano que pasa por los puntos $(3,2,-1)$ $(1,-1,2)$ que es paralelo a la recta $\ell = (1,-1,0) + t(3,2,-2)$

\end{enumerate}




\end{document}
