\documentclass[letterpaper,11pt]{article}
\usepackage[utf8]{inputenc}
\usepackage[spanish,mexico]{babel}
\usepackage{graphicx}
\usepackage{amsmath}
\usepackage{amsthm}
\usepackage{amsfonts}
%\usepackege{amssymb}
\usepackage[hmargin=1in, vmargin=1in]{geometry}
\usepackage{fancyhdr}
\pagestyle{fancy}
\usepackage{tasks}
\lhead{\ExiCarrera}
\chead{\ExiMateria}
\rhead{\ExiParcial}
\cfoot{\ExiEscuela}
\renewcommand{\headrulewidth}{0.4pt}
\renewcommand{\footrulewidth}{0.4pt}

\providecommand{\abs}[1]{\lvert#1\rvert}
\providecommand{\norm}[1]{\lVert#1\rVert}

%=================================================================================
%	Definición de comandos
%=================================================================================
\newcommand{\informacion}[1]{
\begin{center}
\fbox{\fbox{\parbox{\textwidth}{{\footnotesize#1}}}}
\end{center}
\vspace{5mm}}
\newcommand{\datos}{\makebox[0.7\textwidth]{Nombre:~\hrulefill} Fecha:~\hrulefill}
\newcommand{\pregunta}[2]{\item{#2}~{(#1 puntos)}\\ \vspace{5mm}						       
			{\bf Solución}}														  
%	Los siguientes comandos hay que definirlos desde aquí.								    %
\newcommand{\ExiCarrera}{Matemáticas para las Ciencias II.}											  %%%
\newcommand{\ExiMateria}{\textbf{Tarea I}}														%%%%%%%%%%%%%%%%%%%%%%%%%%%%%
\newcommand{\ExiParcial}{Entrega: 7 de marzo de 2020}														  %%%
\newcommand{\ExiEscuela}{\textbf{Facultad de Ciencias, UNAM}}	                		    %

%=================================================================================
%
%=================================================================================
\begin{document}



%%%%%%%%%%%%%%%%%%%%%            CARÁTULA            %%%%%%%%%%%%%%%%%%%%%%%%%
\setlength{\unitlength}{1cm}
\thispagestyle{empty}
\begin{picture}(18,4)
\put(-0.5,1.2){\includegraphics[scale=.25]{unam1.png}}
\put(13.5,1){\includegraphics[scale=.35]{fciencias1.png}}
\end{picture}

\begin{center}
\vspace{-134pt}
\textbf{\large Matemáticas para las Ciencias I}\\[0.2cm]
\textbf{ Semestre 2020-2}\\[0.2cm]
Prof. Pedro Porras Flores\\[0.2cm]
Ayud. Irving Hernández Rosa \\ [0.2cm]
\textbf{Tarea I}
\end{center}
\vspace{-10pt}
\rule{17cm}{0.3mm}
\begin{flushright}
\vspace{-3pt}
\end{flushright}

%%%%%%%%%%%%%%%%%%%%%%%%%%%%%%%%%%%%%%%%%%%%%%%%%%%%%%%%%

\noindent Realice los siguientes ejercicios, escribiendo el procedimiento claramente. Y recuerden que la tarea se entrega en equipos de a lo más tres integrantes. 


\begin{enumerate}


% -----------------------------------------------------
% Problema diez
% -----------------------------------------------------
% -----------------------------------------------------
% Problema once
% -----------------------------------------------------
 
\item Sean $\beta$ y $\gamma$ las bases estándar para $\mathbb{R}^{n}$ y $\mathbb{R}^{m}$ respectivamente. Para cada transformación lineal $T: \mathbb{R}^{n} \longrightarrow \mathbb{R}^{m}$ encontrar su representación matricial.
\begin{tasks}(1)
\task $T: \mathbb{R}^{2} \longrightarrow \mathbb{R}^{3}$ definido por $T(a_{1},a_{2}) = (2a_{1} - a_{2}, 3a_{1} + 4a_{2},a_{1})$
\task $T: \mathbb{R}^{3} \longrightarrow \mathbb{R}^{2}$ definido por $T(a_{1},a_{2},a_{3}) = (2a_{1} + 3a_{2} - a_{3}, a_{1} + a_{3})$
\task $T: \mathbb{R}^{3} \longrightarrow \mathbb{R}$ definido por $T(a_{1},a_{2},a_{3}) = 2a_{1} + a_{2} - 3a_{3}$
\task $T: \mathbb{R}^{3} \longrightarrow \mathbb{R}^{3}$ definido por $T(a_{1},a_{2},a_{3}) = (2a_{2} + a_{3}, -a_{1} + 4a_{2} + 5a_{3},a_{1} + a_{3})$
\end{tasks}

\newpage
% -----------------------------------------------------
% Problema doce
% -----------------------------------------------------
\item Para cada uno de los siguientes pares de bases $\beta$ y $\beta'$ para $\mathbb{R}^{2}$, encuentra la matriz de cambio de coordenadas que cambia las coordenadas de $\beta'$ en las de $\beta$.

\begin{tasks}(1)
\task $\beta = \lbrace \hat{e}_{1},\hat{e}_{2} \rbrace$ y $\beta' = \lbrace (a_{1},a_{2}),(b_{1},b_{2}) \rbrace$
\task $\beta = \lbrace (-1,3),(2,-1) \rbrace$ y $\beta' = \lbrace (0,10),(5,0) \rbrace$
\task $\beta = \lbrace (2,5),(-1,-3) \rbrace$ y $\beta' = \lbrace e_{1},e_{2} \rbrace$
\task $\beta = \lbrace (-4,3),(2,-1) \rbrace$ y $\beta' = \lbrace (2,1),(-4,1) \rbrace$
\end{tasks}

% -----------------------------------------------------
% Problema trece
% -----------------------------------------------------
\item Encontrar la matriz inversa por el método de Gauss-Jordan de las siguientes matrices 

\begin{tasks}(5)
\task $\begin{pmatrix} 1 & 2 \\ 1 & 1 \end{pmatrix}$     
\task $\begin{pmatrix} 1 & 2 \\ 2 & 4 \end{pmatrix}$ 
\task $\begin{pmatrix} 1 & 2 & 1 \\ 1 & 3 & 4 \\ 2 & 3 & -1 \end{pmatrix}$
\task $\begin{pmatrix} 0 & -2 & 4 \\ 1 & 1 & -1 \\ 2 & 4 & -5 \end{pmatrix}$
\end{tasks}


% -----------------------------------------------------
% Problema catorce
% -----------------------------------------------------

\item Calcular el determinante de las siguientes matrices 
\begin{tasks}(4)
\task $\begin{pmatrix} 6 & -3 \\ 2 & 4 \end{pmatrix}$
\task $\begin{pmatrix} -5 & 2 \\ 6 & 1 \end{pmatrix}$ 
\task $\begin{pmatrix} 0 & 1 & 2 \\ -1 & 0 & -3 \\ 2 & 3 & 0 \end{pmatrix}$ 
\task $\begin{pmatrix} 0 & 2 & 1 & 3 \\ 1 & 0 & -2 & 2 \\ 3 & -1 & 0 & 1 \\ -1 & 1 & 2 & 0 \end{pmatrix}$ 
\end{tasks}
% ---------------------------------------------------
% Problema quince
% ----------------------------------------------------
\item Para cada par de vectores $u$ y $v$ en $\mathbb{R}^2$, calcula el área del paralelogramo determinado por $u$ y $v$.

\begin{tasks}(3)
\task $\vec{u} = (3,-2)$ y $\vec{v} = (2,5)$
\task $\vec{u}  = (1,3)$ y $\vec{v}  = (-3,1)$ 
\task $\vec{u}  = (4,-1)$ y $\vec{v}  = (-6,-2)$
\end{tasks}
% ---------------------------------------------------
% Problema diesciseis 
% ----------------------------------------------------
\item Para cada una de las siguientes matrices $A \in M_{nxn}(\mathbb{R})$ determine los valores propios de $A$ y para cada valor propio $\lambda$ de $A$, encontrar el conjunto de vectores propios correspondientes a $A$

\begin{tasks}(3)
\task $A = \begin{pmatrix} 1 & 2 \\ 3 & 2 \end{pmatrix}$
%\task $A = \begin{pmatrix} 2 & 2 \\ 2 & -1 \end{pmatrix}$
\task $A = \begin{pmatrix} 1 & 0 & 2 \\ -1 & 1 & 1 \\ 2 & 0 & 1 \end{pmatrix}$
\task $A = \begin{pmatrix} 0 & -2 & -3 \\ -1 & 1 & -1 \\ 2 & 2 & 5 \end{pmatrix}$
\end{tasks}

% ---------------------------------------------------
% Problema siete
% ----------------------------------------------------

\item Encuentre los ejes principales de la siguiente superficie $x^2 + 4y^2 + 5z^2 + 8xz - 36 = 0$ y con ello construya una matriz de rotación de tal manera que los ejes principales de la superficie coincidan con los con los vectores canónicos de $\mathbb{R}^3$. ¿Qué superficie es?   

\item Aplique el proceso de Gram-Schmidt para construir una base ortonormal en cada caso

\begin{tasks}(2)
\task  $\lbrace (1,1), (1,2) \rbrace$
\task $\lbrace (3,-3), (3,1) \rbrace$
\task $\lbrace (1,-1,-1), (0,3,3), (3,2,4) \rbrace$
\task $\lbrace (1,1,1), (1,1,0), (1,0,0) \rbrace$
\end{tasks}

\item Encuentre la distancia de punto $(2,1,-1)$ al plano $x -2y + 2z + 5 = 0$

\item Encuentre la ecuación del plano que pasa por los puntos $(3,2,-1)$ $(1,-1,2)$ que es paralelo a la recta $\ell = (1,-1,0) + t(3,2,-2)$

\end{enumerate}




\end{document}
