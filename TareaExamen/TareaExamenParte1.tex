\documentclass[letterpaper]{article}
\usepackage[utf8]{inputenc}
\usepackage[spanish]{babel}
\usepackage{amssymb, amsmath}
\usepackage{graphicx}
\usepackage{lipsum}
\usepackage{dsfont}
\usepackage[margin=1.3cm,
vmargin={1.3cm,1.3cm},
includefoot]{geometry}
\usepackage{setspace}
\usepackage{subcaption}
\usepackage{tocloft}
\usepackage{upgreek}
\usepackage{amsthm}
\usepackage{graphicx}
\usepackage{paralist}
\usepackage{fancyhdr}
\usepackage{lmodern}
\usepackage{tcolorbox}
\usepackage{color}
\usepackage{tikz}
\tcbuselibrary{skins,breakable}
\pagestyle{fancy}

\renewcommand{\headrulewidth}{0pt}
\renewcommand{\footrulewidth}{0.4pt}
\cfoot{\textbf{Facultad de Ciencias, UNAM}\\ \thepage}

\newcommand{\V}{\mathds{V}}

\newcommand{\W}{\mathds{W}}

\newcommand{\F}{\mathds{F}}

\newcommand{\tq}{ \quad \cdot  \backepsilon \cdot \quad }

\newcommand{\ld}{\lim\limits_{x \to 0^{+}}}

\newcommand{\li}{\lim\limits_{x \to 0^{-}}}

\newcommand{\la}{\lim\limits_{x \to a}}

\newcommand{\R}{\mathds{R}}

\renewcommand{\*}{\cdot}

\newcommand{\Iden}{\begin{pmatrix}
		1 & 0 & 0\\
		0 & 1 & 0\\
		0 & 0 & 1 
\end{pmatrix}}

\newcommand{\ExiEscuela}{\textbf{Facultad de Ciencias, UNAM}}

\newtcolorbox{ejercicio}[1]{beamer,colback=white!90!white, colframe=black, title=Ejercicio #1}

\makeatletter
\renewcommand*\env@matrix[1][*\c@MaxMatrixCols c]{%
	\hskip -\arraycolsep
	\let\@ifnextchar\new@ifnextchar
	\array{#1}}
\makeatother

\newtheorem{theorem}{Teorema}[section]
\theoremstyle{definition}
\newtheorem{definition}{Definición}

\begin{document}
\begin{center}
	\textbf{\large Matemáticas para las Ciencias II}\\
	\textbf{ Semestre 2020-1}\\
	Prof. Pedro Porras Flores\\
	Ayud. Irving Hérnandez Rosas \\
	Merino Peña Kevin Ariel\\ 317031326\\
	\textbf{Proyecto I}
\end{center}
\rule{17cm}{0.3mm}

\noindent Realice los siguientes ejercicios, escribiendo el procedimiento claramente. Y recuerden que estos proyectos se entregan de manera individual en la plataforma de google classroom. 

\begin{enumerate}
	
	% -----------------------------------------------------
	% Problema uno
	% -----------------------------------------------------
	
	\item  De la definición de parábola deduzca de manera análoga como lo hicimos en la video-clase la ecuación para una parábola cuyo foco se encuentra en el eje $x$, es decir $$y^2 = 4px.$$ 
	
	\begin{definition}
		El conjunto de los puntos del plano $ \tq $ que están a la misma distancia de una recta dada $ D $ y de un punto $ \vec{F} $, que no esté sobre $ D $, recibe el nombre de \textbf{parábola}
	\end{definition}
	Para deducir la ecuación de la parábola supongamos que la coordenadas de $ \vec{F} = (p,0) $ y que la recta D está descrita por $ w = (-p,y) \quad \forall y \in \R $, Luego por la definición que hemos tomado de prábola tenemos que 
	\begin{align*}
	||\vec{u} - \vec{F} || &= || \vec{u} - \vec{w} || && \text{Por definición de distancia} \\
	||(x,y) - (p,0) || &= || (x,y) - (-p,y) || && \text{Los valores de dichos vectores  } \\
	||(x -p,y)|| &= || (x +p,y-y)|| && \text{Restando de manera directa} \\
	\sqrt{<(x -p,y),(x -p,y)>} &= \sqrt{<(x +p,0),(x +p,0)>} && \text{Definición de la norma en vectores } \\
	\sqrt{(x -p)^2 +y^2} &= \sqrt{(x +p)^2 +(0)^2} && \text{Calculando el producto interior  } \\
	(x -p)^2 +y^2 &= (x +p)^2 +(0)^2 && \text{Factorizando } \\
	x^2 -2xp +p^2 +y^2 &= x^2 +2xp +p^2 && \text{Por distributividad  } \\
	y^2 &= 4xp && \text{Agrupando y sumando términos semejantes}
	\end{align*}
	
	\begin{figure}[h!]
		\centering
		\caption{Parábola con foco sobre el eje $x$.}
		\label{F1}
	\end{figure}
	
	% -----------------------------------------------------
	% Problema dos
	% -----------------------------------------------------
	\newpage
	\item De igual manera que se hizo en clase deduzca la ecuación de una elipse cuyos focos se encuentran sobre el eje $y$, esto es: $$\dfrac{x^2}{b^2} + \dfrac{y^2}{a^2}=1$$
	
	\begin{figure}[h!]
		\centering
		\caption{Elipse con focos sobre el eje $y$.}
		\label{F1}
	\end{figure}
	
	% -----------------------------------------------------
	% Problema tres
	% -----------------------------------------------------
	
	
	\item Deduzca la ecuación de la hipérbola de la definición, sin importar donde estén los focos, es decir ya sea que muestre: $$\dfrac{x^2}{a^2} -  \dfrac{y^2}{b^2} = 1 \text{ o }  \dfrac{x^2}{b^2} - \dfrac{y^2}{a^2}=1$$
	
	\begin{figure}[h!]
		\centering
		\caption{Hipérbola con focos sobre el eje $x$.}
		\label{F1}
	\end{figure}
	
	
	
	
\end{enumerate}

\end{document}