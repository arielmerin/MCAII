\documentclass[letterpaper]{article}
\usepackage[utf8]{inputenc}
\usepackage[spanish]{babel}
\usepackage{amssymb, amsmath}
\usepackage{graphicx}
\usepackage{lipsum}
\usepackage{dsfont}
\usepackage[margin=1.5cm,
vmargin={1.5cm,1.3cm},
includefoot]{geometry}
\usepackage{setspace}
\usepackage{subcaption}
\usepackage{tocloft}
\usepackage{upgreek}
\usepackage{amsthm}
\usepackage{graphicx}
\usepackage{paralist}
\usepackage{fancyhdr}
\usepackage{lmodern}
\usepackage{tcolorbox}
\usepackage{color}
\usepackage{tikz}
\tcbuselibrary{skins,breakable}
\pagestyle{fancy}

\renewcommand{\headrulewidth}{0.4pt}
\renewcommand{\footrulewidth}{0.4pt}

\providecommand{\abs}[1]{\lvert#1\rvert}
\providecommand{\norm}[1]{\lVert#1\rVert}														  
\newcommand{\V}{\mathds{V}}

\newcommand{\W}{\mathds{W}}

\newcommand{\F}{\mathds{F}}

\newcommand{\tq}{ \quad \cdot  \backepsilon \cdot \quad }

\newcommand{\ld}{\lim\limits_{x \to 0^{+}}}

\newcommand{\li}{\lim\limits_{x \to 0^{-}}}

\newcommand{\la}{\lim\limits_{x \to a}}

\newcommand{\R}{\mathds{R}}

\newcommand{\Po}{\mathds{P}_2(\mathds{R})}

\renewcommand{\*}{\cdot}

\newcommand{\Iden}{\begin{pmatrix}
		1 & 0 & 0\\
		0 & 1 & 0\\
		0 & 0 & 1 
\end{pmatrix}}
\newcommand{\T}{\begin{pmatrix}
		1 & 3 & 9 \\
		1 & 3 & 4 \\
		0 & 0 & 2 
\end{pmatrix} }

\makeatletter
\renewcommand*\env@matrix[1][*\c@MaxMatrixCols c]{%
	\hskip -\arraycolsep
	\let\@ifnextchar\new@ifnextchar
	\array{#1}}
\makeatother

\newtheorem{theorem}{Teorema}[section]
\theoremstyle{definition}
\newtheorem{definition}{Definición}

\begin{document}
	
	\setlength{\unitlength}{1cm}
	\thispagestyle{empty}
	\begin{picture}(19,3)
	\put(-0.5,1.2){\includegraphics[scale=.20]{unam1.png}}
	\put(16,1){\includegraphics[scale=.29]{fciencias1.png}}
	\end{picture}
	
	\begin{center}
		\vspace{-114pt}
		\textbf{\large Matemáticas para las Ciencias II}\\
		\textbf{ Semestre 2020-2}\\
		Prof. Pedro Porras Flores\\
		Ayud. Irving Hernández Rosas \\
		\textbf{Proyecto III}\\[0.2cm]
		Kevin Ariel Merino Peña\footnote{317031326}\\ [0.2cm]
	\end{center}
	\vspace{-10pt}
	\rule{19cm}{0.3mm}


\noindent Realice los siguientes ejercicios, escribiendo el procedimiento claramente. Y recuerden que estos proyectos se entregan de manera individual en la plataforma de google classroom. \\[0.5cm]
1.  Muestre que los siguientes conjuntos del plano son abiertos: 


$$\text{a) } A = \{ (x,y) \in \mathbb{R} \vert - 1 < x < 1, - 1 < y < 1 \}$$

$$\text{b) }B = \{ (x,y) \in \mathbb{R} \vert \,  0 < y  \}$$

$$\text{c) }A = \{ (x,y) \in \mathbb{R} \vert \, 2 < x^2  + y^2 <  4 \}$$\\[0.5cm]
2.  Calcule los siguientes. límites si existen: 


$$\text{a) }\displaystyle\lim_{(x,y) \to (0,0)} \dfrac{\cos(xy) - 1}{x^2y^2}$$

$$\text{b) }\displaystyle\lim_{(x,y) \to (0,0)} \dfrac{\sin(xy) }{xy}$$

$$\text{c) }\displaystyle\lim_{x \to 1} (x^2 , e^x) $$\\[0.5cm]
3.  Usando la formulación $\epsilon$-$\delta$ muestre: \\

 $$ \text{a) } \displaystyle\lim_{(x,y,z) \to (0,0,0)} \dfrac{xyz}{x^2 + y^2 + z^2} = 0$$\\

$$\text{b) }\displaystyle\lim_{(x,y) \to (0,0)} \dfrac{xy }{\sqrt{x^2 + y^2}} = 0$$\\

$$\text{c) }\displaystyle\lim_{x \to 2} (3x , x^2) = (6,4) $$\\[0.5cm]
4.  Sea $f: \mathbb{R}^2  \longrightarrow \mathbb{R}$ tal que $f(x,y) = \left\{
     \begin{array}{cl}
       \dfrac{x^2y}{\vert x \vert^3 + y^2} & : \text{ si } (x,y) \neq (0,0)\\
       0 & : \text{ si } (x,y) = (0,0)\\
     \end{array}
   \right.$. \\Muestre que $f$ es continua en $(0,0)$






\end{document}
