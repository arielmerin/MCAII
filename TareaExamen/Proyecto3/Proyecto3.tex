\documentclass[letterpaper]{article}
\usepackage[utf8]{inputenc}
\usepackage[spanish]{babel}
\usepackage{amssymb, amsmath}
\usepackage{graphicx}
\usepackage{lipsum}
\usepackage{dsfont}
\usepackage[margin=1.5cm,
vmargin={1.5cm,1.3cm},
includefoot]{geometry}
\usepackage{setspace}
\usepackage{subcaption}
\usepackage{tocloft}
\usepackage{upgreek}
\usepackage{amsthm}
\usepackage{graphicx}
\usepackage{paralist}
\usepackage{fancyhdr}
\usepackage{lmodern}
\usepackage{tcolorbox}
\usepackage{color}
\usepackage{tikz}
\usepackage{wasysym}
\usepackage{textgreek, marvosym}
\tcbuselibrary{skins,breakable}
\pagestyle{fancy}

\renewcommand{\headrulewidth}{0.4pt}
\renewcommand{\footrulewidth}{0.4pt}

\providecommand{\abs}[1]{\left|#1\right|}
\providecommand{\norm}[1]{\left|\left|#1\right|\right|}														  
\newcommand{\V}{\mathds{V}}

\newcommand{\W}{\mathds{W}}

\newcommand{\F}{\mathds{F}}

\newcommand{\tq}{ \quad \cdot  \backepsilon \cdot \quad }

\newcommand{\ld}{\lim\limits_{x \to 0^{+}}}

\newcommand{\li}{\lim\limits_{x \to 0^{-}}}

\newcommand{\la}{\lim\limits_{x \to a}}

\newcommand{\R}{\mathds{R}}

\newcommand{\Po}{\mathds{P}_2(\mathds{R})}

\renewcommand{\*}{\cdot}

\newcommand{\Iden}{\begin{pmatrix}
		1 & 0 & 0\\
		0 & 1 & 0\\
		0 & 0 & 1 
\end{pmatrix}}
\newcommand{\T}{\begin{pmatrix}
		1 & 3 & 9 \\
		1 & 3 & 4 \\
		0 & 0 & 2 
\end{pmatrix} }

\makeatletter
\renewcommand*\env@matrix[1][*\c@MaxMatrixCols c]{%
	\hskip -\arraycolsep
	\let\@ifnextchar\new@ifnextchar
	\array{#1}}
\makeatother

\newtheorem{theorem}{Teorema}[section]
\theoremstyle{definition}
\newtheorem{definition}{Definición}

\begin{document}
	
	\setlength{\unitlength}{1cm}
	\thispagestyle{empty}
	\begin{picture}(19,3)
	\put(-0.5,1.2){\includegraphics[scale=.20]{unam1.png}}
	\put(16,1){\includegraphics[scale=.29]{fciencias1.png}}
	\end{picture}
	
	\begin{center}
		\vspace{-114pt}
		\textbf{\large Matemáticas para las Ciencias II}\\
		\textbf{ Semestre 2020-2}\\
		Prof. Pedro Porras Flores\\
		Ayud. Irving Hernández Rosas \\
		\textbf{Proyecto III}\\[0.2cm]
		Kevin Ariel Merino Peña\footnote{317031326}\\ [0.2cm]
	\end{center}
	\vspace{-10pt}
	\rule{19cm}{0.3mm}


\noindent Realice los siguientes ejercicios, escribiendo el procedimiento claramente. Y recuerden que estos proyectos se entregan de manera individual en la plataforma de google classroom. \\[0.5cm]
1.  Muestre que los siguientes conjuntos del plano son abiertos: 
\begin{definition}
	Sea $ \vec{x}_0  \in \R^n$  y sea $ r \in \R^+ $, la \textbf{bola} de radio r y centro en $ \vec{x}_0 $ es definida por el conjunto de todos los puntos $ \vec{x} $ tal que $ \norm{\vec{x} - \vec{x}_0} < r $.\\ Este conjunto es denotado como $ Br(\vec{x}_0) $ es el conjunto de los puntos $ \vec{x} \in \R^n $ cuya distancia de $ \vec{x}_0 $ es menor que $ r $
\end{definition}

\begin{definition}
	Sea $ U \subset \R^n $. Decimos que $ U $ es \textbf{conjunto abierto} si para cada $ \vec{x}_0 $, existe algún $ r>0 $ tal que $ Br(\vec{x}_0) $ está totalmente contenida en $ U $, $ Br(\vec{x}_0) \subset U $

\end{definition}

$\text{a) } A = \{ (x,y) \in \mathbb{R}^2 \vert - 1 < x < 1, - 1 < y < 1 \}$
\begin{figure}[h!]
	\centering
	\includegraphics[width=0.3\textwidth]{1a}
	\caption{En el plano sólo la intersección es el conjunto A}
\end{figure}
Sea $ \vec{u} \in A $, entonces dicho vector tiene la siguiente forma $\vec{u} = (x_0, y_0)  $, además por ser elemento de $ A $ sus coordendas cumplen lo siguiente
\begin{align*}
	-1 &< x_0 < 1 \label{eq:desigualdadX} \tag{ \saturn } \\
	-1 &< y_0 < 1 \label{eq:desigualdadY} \tag{ \jupiter}
\end{align*}
Tomemos en cuenta los siguientes valores que serán de ayuda para calcular el radio de la Bola abierta
 \begin{align*}
 	r_x &= min\{ 1-x_0, 1+x_0 \} \label{eq:radioX} \tag{ \textdelta}\\
 	r_y &= min\{ 1-y_0, 1+y_0 \} \label{eq:radioY} \tag{\textxi}
 \end{align*}
Lyego, de $ \ref{eq:desigualdadX} $ tenemos que
\begin{align*}
	x_0 < 1 &\implies 0 < 1 - x_0 && \text{Tomando la segunda parte de la desigualdad} \\
	-1 < x_0 &\implies 0 < 1 - x_0 &&\text{Tomando la primera parte de la desigualdad}
\end{align*}
y por $ \ref{eq:desigualdadY} $ se obtiene
\begin{align*}
	y_0 < 1 &\implies 0 < 1 - y_0 && \text{Tomando la segunda parte de la desigualdad} \\
	-1 < y_0 &\implies 0 < 1 - y_0 &&\text{Tomando la primera parte de la desigualdad}
\end{align*}
Por lo tanto, decidimos tomar el radio de la bola como $ r = min\{ r_x, r_y\} $ así aseguraremos que siempre se encuentre dentro del conjunto $ A $-\\


Ahora mostremos que $ B_r(\vec{u})  $ está contenida en $ A $, para ello sea $ \vec{x} \in B_r(\vec{u}) $ entonces dicho vector tendrá coordenadas digamos $ \vec{x} = (x, y) $
\begin{align*}
	\norm{\vec{x} - \vec{u} } &< r &&\text{Por la definición del radio dichos puntos distan menos que r} \\
	\sqrt{(x - x_0)^2 + (y - y_0)^2} &< r &&\text{Empleando la definición de norma}\\
	(x - x_0)^2 + (y - y_0)^2 &< r^2 &&\text{Pues la raíz es continua}
\end{align*}
De la implicación anterior podemos concluir 
\begin{align*}
	(x - x_0)^2 &< r^2 && \text{Por lo anterior y propiedad de la desigualdad en sumas}\\
	\sqrt{(x - x_0)^2} &< \sqrt{r^2} && \text{Pues la raíz es continua}\\
	\abs{x - x_0} &< r && \text{Definición de valor absoluto}\\
	\abs{x - x_0} &< r \leq r_x && \text{Construcción de }r_x\\
	\abs{x - x_0} &< r_x && \text{Transitividad de la desigualdad }\\
	-r_x < x - x_0 &< r_x && \text{Teorema de la desigualdad } \label{eq:distanX}\tag{\Denarius}
\end{align*}
y también observemos que
\begin{align*}
	r_x &\leq 1 - x_0  &1 + x_0 > r_x &&&\text{Por la elección del radio}\\
	r_x &\leq 1 - x_0  &-1 - x_0 \leq -r_x &&&\text{Multiplicando ambos lados de la segunda desigualdad por }-1
\end{align*}
Si juntamos la observación anterior y (\ref{eq:distanX})
\begin{align*}
	-1 - x_0 \leq -r_x &< x - x_0 < r_x < 1 - x_0 &&\text{Empleando ambas desigualdades}\\
	-1 - x_0 &< x - x_0 < 1 - x_0 &&\text{Tomando sólo los valores adecuados}\\
	-1 &< x < 1 &&\text{Sumamos en ambos lados  } x_0\\
\end{align*}
Haciendo lo mismo para $ y $
\begin{align*}
	(y - y_0)^2 &< r^2 && \text{Por lo anterior y propiedad de la desigualdad en sumas}\\
	\sqrt{(y - y_0)^2} &< \sqrt{r^2} && \text{Pues la raíz es continua}\\
	\abs{y - y_0} &< r && \text{Definición de valor absoluto}\\
	\abs{y - y_0} &< r \leq r_y && \text{Construcción de }r_y\\
	\abs{y - y_0} &< r_y && \text{Transitividad de la desigualdad }\\
	-r_y < y - y_0 &< r_y && \text{Teorema de la desigualdad } \label{eq:distanY}\tag{\Florin}
\end{align*}
y también observemos que
\begin{align*}
	r_y &\leq 1 - y_0  &1 + y_0 > r_y &&&\text{Por la elección del radio}\\
	r_y &\leq 1 - y_0  &-1 - y_0 \leq -r_y &&&\text{Multiplicando ambos lados de la segunda desigualdad por }-1
\end{align*}
Si juntamos la observación anterior y (\ref{eq:distanY})
\begin{align*}
	-1 - y_0 \leq -r_y &< y - y_0 < r_y < 1 - y_0 &&\text{Empleando ambas desigualdades}\\
	-1 - y_0 &< y - y_0 < 1 - y_0 &&\text{Tomando sólo los valores adecuados}\\
	-1 &< y < 1 &&\text{Sumamos en ambos lados  } y_0
\end{align*}
\begin{center}
De esta manera, en cualquier caso tenemos que $ \vec{x} \in A $  $\therefore B_r(\vec{u})\subset A $\\
$ \therefore A $ es abierto.	
\end{center}

%%%%%%%%%%%%%%%%%%%%%%%%%%%%%%%%%%%%%%%%%%%%%%%%%%%55
%
%	Segundo Ejercicio, ya hecho ni te apures Ariel
%
%%%%%%%%%%%%%%%%%%%%%%%%%%%%%%%%%%%%%%%%%%%%%%%%%%%55
\noindent$\text{b) }B = \{ (x,y) \in \mathbb{R}^2 \vert \,  0 < y  \}$
\begin{figure}[h]
	\centering
	\includegraphics[height=0.3\textwidth]{1b}
	\caption{En el plano, los cuadrantes 1 y 2, son el conjunto B}
\end{figure}
Sean $ (x,y) \in B $ y $r >0 $, por demostrar $ Br(x,y) \subset B $.\\
La bola de radio más grande que podemos dar es $ r = y $ y como $ (x,y) \in B $, entonces $ y>0 $.\\
Ahora queremos mostrar que
\[ (x_1,y_1)\in Br(x,y) \implies (x_1,y_1)\in B \]  
Sea $ (x_1, y_1)\in Br(x,y) $, entonces

\begin{align*}
	\abs{y_1 - y} &= \sqrt{(y_1 -y)^2} \leq \sqrt{(x_1 - x)^2 + (y_1 - y)^2} \leq r\\
	\abs{y_1 - y} &= \sqrt{(y_1 -y)^2} \leq \sqrt{(x_1 - x)^2 + (y_1 - y)^2} \leq y\\
	\abs{y_1 - y} &\leq y\\
	-y<y_1 - y &< y\\
	0<y_1  &< 2y
\end{align*}
\begin{center}
Por lo tanto $  0< y_1 \implies (x_1, y_1) \in B$, así que $ Br(x,y) \subset B $, \\$ \therefore  $ B es abierto.
\end{center}

%%%%%%%%%%%%%%%%%%%%%%%%%%%%%%%%%%%%%%%%%%%%%%
%
%		TERCER EJERCICIO OREMOS
%
%%%%%%%%%%%%%%%%%%%%%%%%%%%%%%%%%%%%%%%%%%%%%%
\noindent$\text{c) }C = \{ (x,y) \in \mathbb{R}^2 \vert \, 2 < x^2  + y^2 < 4\}$
\begin{figure}[h]
	\centering
	\includegraphics[width=0.3\textwidth]{1c}
	\caption{En el plano sólo la parte azul es el conjunto C}
\end{figure}

Sea $ \vec{p_0} = (x_0, y_0) \in C  $, que $ \vec{p_0} $ esté en el conjunto $ C $ eso es 
\begin{align*}
	2 &< \norm{\vec{p_0}}^2 < 4 &&\text{ Porque está en el conjunto C}\\
	\sqrt{2} &< \norm{\vec{p_0}} < 2 &&\text{ Sacando la raíz en todos lados de la desigualdad}\\
\end{align*} 
Entonces tomemos el radio de la bola como 
\[ 0 < r < min \{ \norm{\vec{p_0}} - \sqrt{2}, 2- \norm{\vec{p_0}} \} \]
y tomemos $ \vec{q} \in B_r(\vec{p_0}) $ ahora mostremos que $ B_r(\vec{p_0}) \subset C $ para ello, basta ver que $ \vec{q} \in C $, esto es $ \sqrt{2} < \norm{\vec{q}} < 2 $.\\
Observemos que 
\begin{align*}
	\norm{\vec{q}} - \norm{\vec{p_0}} &< \norm{\vec{q} - \vec{p_0}} &&\text{Por definición, debe distar menos que ello}\\
\end{align*}
Como $ \vec{q}\in B_r(\vec{p_0}) $, entonces
\begin{align*}
	\norm{\vec{q} - \vec{p_0}} &< r && \text{Por definición del radio}\\
	\norm{\vec{q}} - \norm{\vec{p_0}} \leq \norm{\vec{q} - \vec{p_0}} &< r && \text{Por la desigualdad del triángulo}\\
	\norm{\vec{q}} - \norm{\vec{p_0}} \leq \norm{\vec{q} - \vec{p_0}} &< r < 2 - \norm{\vec{p_0}} && \text{Por la conición anterior}\\
	\norm{\vec{q}} - \norm{\vec{p_0}} &<  2 - \norm{\vec{p_0}} && \text{Tomando una parte de la inequidad}\\
	\norm{\vec{q}}  &< 2&& \text{Sumando en todos lados } \norm{\vec{p_0}} \\
	\norm{\vec{q}} &< 2&& \text{Así el punto cumple la condición de estar en } C\\
\end{align*}
En otro caso, si pasa que $ r < \norm{\vec{p_0}} - \sqrt{2} $, entonces tendríamos
\begin{align*}
	\norm{\vec{q} - \vec{p_0}} &< r && \text{Por definición del radio}\\
	\norm{\vec{q}} - \norm{\vec{p_0}} \leq \norm{\vec{q} - \vec{p_0}} &< r && \text{Por la desigualdad del triángulo}\\
	\norm{\vec{q}} - \norm{\vec{p_0}} \leq \norm{\vec{q} - \vec{p_0}} &< r < \norm{\vec{p_0}} - \sqrt{2} && \text{Por la conición anterior}\\
	\norm{\vec{q}} - \norm{\vec{p_0}} &<  \norm{\vec{p_0}} - \sqrt{2} && \text{Tomando una parte de la inequidad}\\
	\norm{\vec{q}}  &< \sqrt{2}&& \text{Sumando en todos lados } 		\norm{\vec{p_0}} \\
	\norm{\vec{q}} &< \sqrt{2}&& \text{Así el punto cumple la condición de estar en } C\\
\end{align*}
\begin{center}
	Así, hemos visto que en cualquier caso $ \vec{q} \in C $ por lo tanto $ B_r(\vec{p_0})  \subset C$\\
	$ \therefore C $ es un conjunto abierto
\end{center}


\noindent2.  Calcule los siguientes. límites si existen: 
\begin{definition}
	Sea $ f: A \subset \R^n \to \R^m $ y $ \vec{x}_0 \in A $, un punto de acumulación de $ A $. Entonces se dice qe el límite de $  f(\vec{x}) $, cuando $ \vec{x} $ tiende a $ \vec{x}_0 $, es $ \vec{\textit{l}} \in \R^m $ y se denota
	\[ \lim\limits_{\vec{x} \to \vec{x}_0} = \vec{\textit{l}} \qquad \text{ o } \qquad f(\vec{x}) \to \vec{\textit{l}} \]
	Si $ \forall \epsilon > 0, \quad \exists  \delta > 0 $ tal que $ \vec{x} \to \vec{x}_0 $
\end{definition}


\noindent$\text{a) }\displaystyle\lim_{(x,y) \to (0,0)} \dfrac{\cos(xy) - 1}{x^2y^2}$\\
Resultará conveniente recordar de nuestro curso de Matemáticas para las ciencias aplicadas I que $$ \lim\limits_{\alpha \to 0} \dfrac{cos(\alpha) - 1}{\alpha^2} = -\dfrac{1}{2} $$.  
Entonces tomemos el siguiente cambio de variable $ \alpha = xy $
y por el recordatorio anterior, tenemos que 
 
$$\displaystyle\lim_{(x,y) \to (0,0)} \dfrac{\cos(xy) - 1}{x^2y^2} = -\dfrac{1}{2}$$

\noindent$\text{b) }\displaystyle\lim_{(x,y) \to (0,0)} \dfrac{\sin(xy) }{xy}$\\
Para este segundo ejercicio, tomemos un cambio de variable $ y =\sqrt{r}  $ y $ x = \sqrt{r} $ entonces $ xy = r $ por lo que
$$\text{b) }\displaystyle\lim_{(r\to 0} \dfrac{\sin(r) }{r}$$
Y de nuestro curso de Matemáticas para las ciencias aplicadas I tenemos que $ \lim\limits_{\alpha \to 0} \dfrac{sin(\alpha)}{\alpha} = 1 $, por lo tanto
$$\displaystyle\lim_{(x,y) \to (0,0)} \dfrac{\sin(xy) }{xy} = 1$$

$\text{c) }\displaystyle\lim_{x \to 1} (x^2 , e^x) $\\[0.5cm]
Tenemos un teorema enunciado en clase sobre las propiedades de los límites, una de ellas dice:
\begin{definition}
	Si $ f(\vec{x}) = (f_1(\vec{x}), \dots, f_m(\vec{x})) $ donde $ f_i:A \to \R  $, $ i \in \{ 1, \dots, m \} $  son las componentes de la función de $ f $, entonces 
	\[ \lim\limits_{\vec{x} \to \vec{x}_0 } f(\vec{x}) = (l_1, l_2, \dots, l_m) \] si y sólo si\[ \lim\limits_{\vec{x} \to \vec{x}_0} f_i(\vec{x}) = l_i \] para cada $ i \in \{ 1, 2, \dots, m \} $
\end{definition}
entonces si
\begin{align*}
	\lim_{x \to 1} (x^2 , e^x) &=\left( \lim_{x \to 1} x^2, \lim_{x \to 1} e^x \right)\\
	\lim_{x \to 1} (x^2 , e^x) &=\left( (1)^2,  e^{1} \right)\\
	\lim_{x \to 1} (x^2 , e^x) &=\left( 1,  e \right)\\
\end{align*}

\noindent3.  Usando la formulación $\epsilon$-$\delta$ muestre: \\

\noindent$ \text{a) } \displaystyle\lim_{(x,y,z) \to (0,0,0)} \dfrac{xyz}{x^2 + y^2 + z^2} = 0$\\

Sea $ \epsilon > 0 $, notemos que 
\begin{align*}
	\norm{\dfrac{xyz}{x^2 + y^2 + z^2} - 0} &= \abs{\dfrac{xyz}{x^2 + y^2 + z^2} } \leq \abs{ \dfrac{xyz}{xy}  }  = \abs{z}\\
	\norm{\dfrac{xyz}{x^2 + y^2 + z^2} - 0} & \leq \sqrt{(z)^2}\\
	\norm{\dfrac{xyz}{x^2 + y^2 + z^2} - 0} & \leq \sqrt{(z)^2}\leq \sqrt{x^2 + y^2 + z^2} \\
\end{align*}
También observemos que $ 0 \leq x^2 \leq x^2 + y^2 + z^2$, entonces $ 0 \leq \dfrac{1}{x^2 + y^2 + z^2} \leq \dfrac{1}{z^2} $.\\
Así, también veamos qe $ \sqrt{x^2 + y^2 + z^2}  = \norm{(x,y,z)} $ entonces
\[ \norm{\dfrac{xyz}{x^2 + y^2 + z^2} - 0} \leq \sqrt{x^2 + y^2 + z^2} \]
\[ \norm{\dfrac{xyz}{x^2 + y^2 + z^2} - 0} \leq \norm{(x,y,z)} \]
Por lo que basta con tomar $ \delta = \epsilon $

$\text{b) }\displaystyle\lim_{(x,y) \to (0,0)} \dfrac{xy }{\sqrt{x^2 + y^2}} = 0$\\
Sea $ \epsilon > 0 $, consideremos 
\begin{align*}
	\norm{\dfrac{xy}{\sqrt{x^2 + y^2}} - 0} &= \abs{\dfrac{xy}{\sqrt{x^2 + y^2}}}\\
	&= \dfrac{\abs{xy}}{\sqrt{x^2 + y^2}}\\
\end{align*}

Por otro lado, veamos que $ 0 \leq \abs{xy} \leq x^2 + y^2 $, entonces $ 0 < \dfrac{\abs{xy}}{\sqrt{x^2 + y^2}} \leq \dfrac{x^2 + y^2}{\sqrt{x^2 + y^2}} $
\begin{align*}
	0 < \dfrac{\abs{xy}}{\sqrt{x^2 + y^2}} &\leq \dfrac{x^2 + y^2}{\sqrt{x^2 + y^2}}\\
	0 < \dfrac{\abs{xy}}{\sqrt{x^2 + y^2}} &\leq \sqrt{x^2 + y^2}\\
	0 < \dfrac{\abs{xy}}{\sqrt{x^2 + y^2}} &\leq \norm{(x,y)} < \epsilon\\
\end{align*}
Entonces sólo basta tomar $ \delta = \epsilon $\\


\noindent$\text{c) }\displaystyle\lim_{x \to 2} (3x , x^2) = (6,4) $\\[0.5cm]
Por lo mencionado anteriormente,  $ \lim_{x \to 2} (3x , x^2) = (6,4) $ puede expresarse como 
\[  (\lim_{x \to 2} 3x , \lim_{x \to 2}x^2) = (6,4)\]
por lo tanto, para el primer caso.\\
\begin{proof}
	Sea $ \epsilon > 0 $, consideremos 
	\begin{align*}
		\abs{f(x) - l} &= \abs{3x - 6}\\
		\abs{f(x) - l} &= 3\abs{x - 2} && \text{Observemos que } \abs{x - a} = \abs{x - 2}\\
	\end{align*}
	Así, basta tomar $ \delta = \dfrac{\epsilon}{2} $
\end{proof}
Luego, para el segundo valor
\begin{proof}
	Sea $ \epsilon > 0 $, veamos que
	\begin{align*}
		\abs{f(x) - l} &= \abs{x^2 - 4}\\
		\abs{f(x) - l} &= \abs{x - 2}\abs{x + 2}\\
	\end{align*}
	Por otra parte, sea $ \delta_0 = 1 $
	\begin{align*}
		\abs{x -2} &< 1\\
		-1< x -2 &< 1\\
		3< x +2 &< 5\\
		\abs{x+2} &< 5\\
	\end{align*}
	De esta manera basta tomar $ \delta = min \left\lbrace 1, \dfrac{\epsilon}{5}\right\rbrace $
\end{proof}
4.  Sea $f: \mathbb{R}^2  \longrightarrow \mathbb{R}$ tal que $f(x,y) = \left\{
     \begin{array}{cl}
       \dfrac{x^2y}{\vert x \vert^3 + y^2} & : \text{ si } (x,y) \neq (0,0)\\
       0 & : \text{ si } (x,y) = (0,0)\\
     \end{array}
   \right.$ \\Muestre que $f$ es continua en $(0,0)$
Para averiguar quién es el límite, tomémonos traectorias distintas. \\
Deifinimos $ y = g(x) = 0 $
\begin{align*}
	\lim\limits_{(x,y)\to (0,0)} f(x,y) &= \lim\limits_{(x,y) \to (0,0)} f(x,g(x))\\
	\lim\limits_{(x,y)\to (0,0)} f(x,y) &= \lim\limits_{(x,y) \to (0,0)} f(x,0)\\
	\lim\limits_{(x,y)\to (0,0)} f(x,y) &= \lim\limits_{x \to 0} \dfrac{x^2y}{\vert x \vert^3 + y^2} \\
	\lim\limits_{(x,y)\to (0,0)} f(x,y) &= \lim\limits_{x \to 0} \dfrac{x^2(0)}{\vert x \vert^3 + (0)^2} \\
	\lim\limits_{(x,y)\to (0,0)} f(x,y) &= 0
\end{align*}
Por otra parte, definamos $ y = g(x) = x$ por lo que
\begin{align*}
	\lim\limits_{(x,y)\to (0,0)} f(x,y) &= \lim\limits_{(x,y) \to (0,0)} f(x,g(x))\\
	\lim\limits_{(x,y)\to (0,0)} f(x,y) &= \lim\limits_{(x,y) \to (0,0)} f(x,x)\\
	\lim\limits_{(x,y)\to (0,0)} f(x,y) &= \lim\limits_{x \to 0} \dfrac{x^3}{\vert x \vert^3 + x^2} \\
	\lim\limits_{(x,y)\to (0,0)} f(x,y) &= \lim\limits_{x \to 0} \dfrac{x^3}{x^2  \left( \dfrac{\abs{x}^3}{x^2} + 1 \right) } \\
	\lim\limits_{(x,y)\to (0,0)} f(x,y) &= \lim\limits_{x \to 0} \dfrac{x}{\dfrac{\abs{x}^3}{x^2} + 1 } \\
	\lim\limits_{(x,y)\to (0,0)} f(x,y) &=  \dfrac{\lim\limits_{x \to 0} x}{\lim\limits_{x \to 0} \left( \dfrac{\abs{x}^3}{x^2} + 1 \right) } \\
	\lim\limits_{(x,y)\to (0,0)} f(x,y) &=  \dfrac{\lim\limits_{x \to 0} x}{\lim\limits_{x \to 0}  \dfrac{3\abs{x}^2 \abs{x}'}{2x} + 1  }  && \text{Aplicando ley de L'Hôpital}\\
	\lim\limits_{(x,y)\to (0,0)} f(x,y) &=  \dfrac{\lim\limits_{x \to 0} x}{ \dfrac{3}{2} \lim\limits_{x \to 0}  \dfrac{\abs{x}^2}{x}\* \lim\limits_{x \to 0} \abs{x}'  + 1  }  && \text{Por la regla del producto}\\
	\lim\limits_{(x,y)\to (0,0)} f(x,y) &=  \dfrac{\lim\limits_{x \to 0} x}{ \dfrac{3}{2} \* 0\{-1,1\}  + 1  }  && \text{Por definición de la derivada del valor absoluto }\\
	\lim\limits_{(x,y)\to (0,0)} f(x,y) &=  \dfrac{\lim\limits_{x \to 0} x}{1}  && \text{}\\
	\lim\limits_{(x,y)\to (0,0)} f(x,y) &=  0 && \text{}\\
\end{align*}
Sea $ \epsilon > 0 $, consideremos $ \norm{f(x,y) - l} $, donde $ f(x,y) = \dfrac{x^2y}{\vert x \vert^3 + y^2} $ y $ l = 0 $, entonces
\begin{align*}
	\norm{\dfrac{x^2y}{\vert x \vert^3 + y^2} - 0} &= \abs{\dfrac{x^2y}{\vert x \vert^3 + y^2}} \leq \abs{\dfrac{x^2y}{x^2}}
\end{align*}
luego, veamos que $ \abs{\dfrac{x^2y}{x^2}} = \abs{y} = \sqrt{y^2} $ y como $ 0 \leq y^2 \leq y^2 + x^2 $ tenemos que
\begin{align*}
	\abs{\dfrac{x^2y}{\abs{x}^3 + y^2}} &\leq \abs{y} = \sqrt{y^2} && \text{Definición de valor absoluto}\\
	\abs{\dfrac{x^2y}{\abs{x}^3 + y^2}} &\leq \sqrt{y^2} && \text{Por la igualdad planteada arriba}\\
	\abs{\dfrac{x^2y}{\abs{x}^3 + y^2}} &\leq \sqrt{y^2} \leq \sqrt{x^2 + y^2}&& \text{Por la observación del inicio}\\
	\abs{\dfrac{x^2y}{\abs{x}^3 + y^2}} &\leq \sqrt{y^2} \leq \sqrt{x^2 + y^2} && \text{Puesto que }\sqrt{x^2 + y^2} = \norm{(x,y)} \\
	\abs{\dfrac{x^2y}{\abs{x}^3 + y^2}} &\leq \sqrt{y^2} \leq \norm{(x,y)} < \epsilon && \text{Por hipótesis} \\
	\abs{\dfrac{x^2y}{\abs{x}^3 + y^2}} &\leq \norm{(x,y)} < \epsilon \\
\end{align*}
Por lo tanto, basta tomar $ \delta = \epsilon $ :)
\end{document}
