\documentclass[letterpaper]{article}
\usepackage[utf8]{inputenc}
\usepackage[spanish]{babel}
\usepackage{amssymb, amsmath}
\usepackage{stackengine}
\usepackage{graphicx}
\usepackage{ mathrsfs }
\usepackage{lipsum}
\usepackage{dsfont}
\usepackage[margin=1.5cm,
vmargin={1.5cm,0.7cm},
includefoot]{geometry}
\usepackage{setspace}
\usepackage{subcaption}
\usepackage{tocloft}
\usepackage{upgreek}
\usepackage{amsthm}
\usepackage{graphicx}
\usepackage{paralist}
\usepackage{fancyhdr}
\usepackage{lmodern}
\usepackage{tcolorbox}
\usepackage{color}
\usepackage{tikz}
\usepackage{wasysym}
\usepackage{textgreek, marvosym}
\tcbuselibrary{skins,breakable}
\pagestyle{fancy}

\renewcommand{\headrulewidth}{0.4pt}
\renewcommand{\footrulewidth}{0.4pt}

\renewcommand{\d}{\partial}

\providecommand{\abs}[1]{\left|#1\right|}
\providecommand{\norm}[1]{\left|\left|#1\right|\right|}														  
\providecommand{\pint}[1]{\langle#1\rangle}														  
\newcommand{\V}{\mathds{V}}

\newcommand{\W}{\mathds{W}}

\newcommand{\F}{\mathds{F}}

\newcommand{\tq}{ \quad \cdot  \backepsilon \cdot \quad }

\newcommand{\ld}{\lim\limits_{x \to 0^{+}}}

\newcommand{\li}{\lim\limits_{x \to 0^{-}}}

\newcommand{\la}{\lim\limits_{x \to a}}

\renewcommand{\l}{\ell}

\newcommand{\R}{\mathds{R}}

\newcommand{\Po}{\mathds{P}_2(\mathds{R})}

\renewcommand{\*}{\cdot}

\newcommand{\Iden}{\begin{pmatrix}
		1 & 0 & 0\\
		0 & 1 & 0\\
		0 & 0 & 1 
\end{pmatrix}}
\newcommand{\T}{\begin{pmatrix}
		1 & 3 & 9 \\
		1 & 3 & 4 \\
		0 & 0 & 2 
\end{pmatrix} }

\makeatletter
\renewcommand*\env@matrix[1][\arraystretch]{%
	\edef\arraystretch{#1}%
	\hskip -\arraycolsep
	\let\@ifnextchar\new@ifnextchar
	\array{*\c@MaxMatrixCols c}}
\makeatother

\newtheorem{theorem}{Teorema}[]
\theoremstyle{definition}
\newtheorem{definition}{Definición}


\begin{document}
	
	\setlength{\unitlength}{1cm}
	\thispagestyle{empty}
	\begin{picture}(19,3)
	\put(-0.5,1.2){\includegraphics[scale=.20]{img/unam1.png}}
	\put(16,1){\includegraphics[scale=.29]{img/fciencias1.png}}
	\end{picture}
	
	\begin{center}
		\vspace{-114pt}
		\textbf{\large Matemáticas para las Ciencias II}\\
		\textbf{ Semestre 2020-2}\\
		Prof. Pedro Porras Flores\\
		Ayud. Irving Hernández Rosas \\
		\textbf{Proyecto V}\\[0.2cm]
		Kevin Ariel Merino Peña\footnote{Número de cuenta 317031326}\\ [0.2cm]
	\end{center}
	\vspace{-10pt}
	\rule{19cm}{0.3mm}
	
\noindent Realice los siguientes ejercicios, escribiendo el procedimiento claramente. Y recuerden que estos proyectos se entregan de manera individual en la plataforma de google classroom.\\


\noindent1.  Verifique el primer caso de la regla de la cadena de la composición $ f\circ \vec{\gamma}$ para cada uno de los siguientes casos, esto es primero haga la composición y derive, y le luego use la regla de la cadena y vea que se llega al mismo resultado.\\
\begin{theorem}[Regla de la cadena]
	\relax
	Sean $ U\subset \R^n $ y $ V \subset \R^m $ conjuntos abiertos, $ g: U \subset \R^n \to \R^m $ y $ f: V\subset \R^m \to \R^p $ dos funciones tales que $ g $ manda a $ U $ en $ V $ \textit{i.e. } $ f \circ g $. Supogamos que $ g $ es diferenciable en $ \vec{x_0} $ y $ D(f \circ g) (\vec{x_0}) = Df(g(\vec{x_0}))Dg(\vec{x_0}) $.\\
	\begin{itemize}
		\item \textbf{Primer caso de la regla de la cadena}\\
		
		Supongamos $ \vec{\gamma}: \R \to \R^3 $ es una trayectoria diferenciable y $ f: \R^3 \to \R $. Sea $ h(t) = f(\vec{\gamma})(t) = f(x(t), y(t), z(t)  $ donde 
		$ \vec{\gamma}(t)= (x(t),y(t),z(t)) $. Entonces
		\[  \dfrac{dh}{dt} = \dfrac{\d f}{\d x} \dfrac{dx}{dt} +  \dfrac{\d f}{\d y} \dfrac{dy}{dt} +  \dfrac{\d f}{\d z} \dfrac{dz}{dt} \]
		esto es:
		\[ \dfrac{dh}{dt} = \nabla f(\gamma(t)) \* \vec{\gamma'}(t) \] donde $ \vec{\gamma'}(t)= (x'(t),y'(t),z'(t))  $.
		
		\item  \textbf{Segundo caso de la regla de la cadena }\\
		
		Sean $ f:\R^3 \to \R $ y $ g:\R^3 \to \R^3 $. Escribimos 
		\[ g(x,y,z) = (u(x,y,x), v(x,y,z), w(x,y,z) \quad \text{  y  }\quad h(x,y,z) = f(u(x,y,z), u(x,y,z), w(x,y,z))  \]
		Entonces:
	\[ \begin{pmatrix}
	\dfrac{\d h}{\d x} & \dfrac{\d h}{\d y} & \dfrac{\d h}{\d x}
	\end{pmatrix} = \begin{pmatrix}
	\dfrac{\d f}{\d x} & \dfrac{\d f}{\d y} & \dfrac{\d f}{\d z}
	\end{pmatrix} \begin{pmatrix}[2]
	\dfrac{\d u}{\d x} & \dfrac{\d u}{\d x} & \dfrac{\d u}{\d x}\\
	\dfrac{\d v}{\d x} & \dfrac{\d v}{\d x} & \dfrac{\d v}{\d x}\\
	\dfrac{\d w}{\d x} & \dfrac{\d w}{\d x} & \dfrac{\d w}{\d x}\\
	\end{pmatrix} \]
	\end{itemize}
\end{theorem}

a) $f(x,y) = xy$, $\vec{\gamma}(t) =(e^t, \cos(t)) $.\\
Tenemos que $  f \circ \gamma(t) = e^t\cos(t) $ y su derivada es 
\begin{align*}
	\dfrac{d}{dt} (f\circ \gamma) &= \dfrac{d}{dt}e^t\cos(t) &&\text{Planteando la derivada}\\
	\dfrac{d}{dt} (f\circ \gamma) &= e^t\dfrac{d}{dt}\cos(t) + \cos(t)\dfrac{d}{dt}e^t &&\text{Por la regla del producto en derivadas}\\
	\dfrac{d}{dt} (f\circ \gamma) &= e^t(- \sin(t)) + \cos(t)e^t &&\text{Por nuestro curso de Cálculo I}\\
	\dfrac{d}{dt} (f\circ \gamma) &= e^t\cos(t) - e^t \sin(t) &&\text{Conmutando la suma de funciones}\\
\end{align*}
por otra parte, por el primer caso de la regla de la cadena, obtenemos
\[ 	\dfrac{d}{dt}(f\circ g) = 	\dfrac{df}{dx}\dfrac{dx}{dt} + \dfrac{df}{dy}\dfrac{dy}{dt} \]
entonces calculemos las siguientes derivadas
\begin{align*}
	\dfrac{\d f}{\d x}(xy) &= y &&\text{Por la regla del producto}\\
	\dfrac{\d f}{\d y}(xy) &= x &&\text{Por la regla del producto}\\
	\dfrac{dx}{dt}(e^t) &= e^t &&\text{Por propiedades de la exponencial}\\
	\dfrac{dy}{dt}(\cos(t)) &= -\sin(t) &&\text{Por características de las trigonométricas}\\
\end{align*}
Así, se tiene que 
\[ \dfrac{d}{dt}(f\circ g) = ye^t - x\sin(t) \] y como $ x = e^t $ y $ y = \cos(t) $
\[ \therefore  \qquad \dfrac{d}{dt}(f\circ g) = \cos(t)e^t-e^t\sin(t) \]


b) $f(x,y) = xy$, $\vec{\gamma}(t) =(3t^2, t^3) $.\\

Tenemos que $  f \circ \gamma(t) = e^{(3t^2)(t^3)} = e^{3t^5}  $ y su derivada es 

\begin{align*}
	\dfrac{d}{dt} (f\circ \gamma) &= \dfrac{d}{dt} e^{3t^5} &&\text{Planteando la derivada}\\
	\dfrac{d}{dt} (f\circ \gamma) &= e^{3t^5}\dfrac{d}{dt} 3t^5 &&\text{Por la regla de la derivada para la exponencial}\\
	\dfrac{d}{dt} (f\circ \gamma) &= e^{3t^5}(15t^4) &&\text{Derivando un monomio}\\
	\dfrac{d}{dt} (f\circ \gamma) &= 15t^4e^{3t^5} &&\text{Derivando un monomio}\\
\end{align*}
por otra parte, por el primer caso de la regla de la cadena, obtenemos
\[ 	\dfrac{d}{dt}(f\circ g) = 	\dfrac{df}{dx}\dfrac{dx}{dt} + \dfrac{df}{dy}\dfrac{dy}{dt} \]
entonces calculemos las siguientes derivadas
\begin{align*}
	\dfrac{\d f}{\d x}(e^{xy}) &= ye^{xy} &&\text{Por la regla de la exponencial}\\
	\dfrac{\d f}{\d y}(e^{xy}) &= xe^{xy} &&\text{Por la regla de la exponencial}\\
	\dfrac{dx}{dt}(3t^2) &= 6t &&\text{Por propiedades de la derivada en exponentes}\\
	\dfrac{dy}{dt}(t^3) &= 3t^2 &&\text{Por propiedades de la derivada en exponentes}\\
\end{align*}
Así, se tiene que 
\[ \dfrac{d}{dt}(f\circ g) = ye^{xy}(6t) - xe^{xy}(3t^2) \] y como $ x = 3t^2 $ y $ y = t^3 $
\[ \therefore  \qquad \dfrac{d}{dt}(f\circ g) = 15t^4e^{3t^5} \]

c) $f(x,y) = (x^2 + y^2)\ln{\sqrt{x^2 + y^2}}$, $\vec{\gamma}(t) =(e^t, e^{-t}) $.\\

Tenemos que $  f \circ \gamma(t) = (e^{2t} + e^{-2t})\ln \sqrt{e^{2t} + e^{-2t}}  $ y su derivada es 

\begin{align*}
\dfrac{d}{dt} (f\circ \gamma) &= \dfrac{d}{dt}((e^{2t} + e^{-2t})\ln \sqrt{e^{2t} + e^{-2t}})  &&\text{Planteando la derivada }\\
\dfrac{d}{dt} (f\circ \gamma) &= \dfrac{d}{dt}((e^{2t} + e^{-2t}) \* \dfrac{1}{2} \ln (e^{2t} + e^{-2t}))  &&\text{Pues } \ln(a^c) = c\* \ln(a)\\
\dfrac{d}{dt} (f\circ \gamma) &= \dfrac{1}{2} \ln (e^{2t} + e^{-2t})\*\dfrac{d}{dt}(e^{2t} + e^{-2t}) +  (e^{2t} + e^{-2t})\*\dfrac{d}{dt}\left(\dfrac{1}{2} \ln (e^{2t} + e^{-2t})\right)  && \text{Por regla del producto en derivadas}\\
\dfrac{d}{dt} (f\circ \gamma) &= \dfrac{1}{2} \ln (e^{2t} + e^{-2t})\*2(e^{2t} - e^{-2t}) +  (e^{2t} + e^{-2t})\*\dfrac{d}{dt}\left(\dfrac{1}{2} \ln (e^{2t} + e^{-2t})\right)  && \text{Derivando la primera parte}\\
\dfrac{d}{dt} (f\circ \gamma) &= \dfrac{1}{2} \ln (e^{2t} + e^{-2t})\*2(e^{2t} - e^{-2t}) +  \dfrac{e^{2t} + e^{-2t} (2e^{2t}-2e^{-2t})}{2\sqrt{e^{2t}+e^{-2t}}\sqrt{e^{2t}+e^{-2t}}} && \text{Derivando la segunda parte}\\
\dfrac{d}{dt} (f\circ \gamma) &= (e^{2t} - e^{-2t})(2\ln\sqrt{e^{2t} + e^{-2t} } + 1) && \text{Factorizando } (e^{2t} - e^{-2t})\\
\end{align*}
por otra parte, por el primer caso de la regla de la cadena, obtenemos
\[ 	\dfrac{d}{dt}(f\circ g) = 	\dfrac{df}{dx}\dfrac{dx}{dt} + \dfrac{df}{dy}\dfrac{dy}{dt} \]
entonces calculemos las siguientes derivadas
\begin{align*}
\dfrac{\d f}{\d x}((x^2 + y^2)\ln{\sqrt{x^2 + y^2}} ) &= x(2\ln\sqrt{x^{2} + y^{2}} + 1)  &&\text{Haciendo la parcial con }x\\
\dfrac{\d f}{\d y}((x^2 + y^2)\ln{\sqrt{x^2 + y^2}}) &= y(2\ln\sqrt{x^2 + y^2} + 1)  &&\text{El caso anterior es homólogo con }y\\
\dfrac{dx}{dt}(e^t) &= e^t &&\text{Por propiedades de la exponencial}\\
\dfrac{dy}{dt}(-e^t) &= -e^{-t} &&\text{Por propiedades de la exponencial }\\
\end{align*}
Así, se tiene que 
\[ \dfrac{d}{dt}(f\circ g) = x(2\ln\sqrt{x^{2} + y^{2}} + 1)\*e^t + y(2\ln\sqrt{x^2 + y^2} + 1)\*(-e^{-t}) \] y como $ x = e^t  $ y $ y = e^{-t} $
\[ \therefore  \qquad \dfrac{d}{dt}(f\circ g) = (e^{2t} - e^{-2t})(2\ln\sqrt{e^{2t} + e^{-2t} } + 1)  \]

d) $f(x,y) = xe^{x^2 + y^2}$, $\vec{\gamma}(t) =(t, -t) $.\\

Tenemos que $  f \circ \gamma(t) = te^{2t^2}   $ y su derivada es 

\begin{align*}
\dfrac{d}{dt} (f\circ \gamma) &= \dfrac{d}{dt} te^{2t^2} &&\text{Planteando la derivada }\\
\dfrac{d}{dt} (f\circ \gamma) &= t\dfrac{d}{dt} e^{2t^2}+e^{2t^2}\*\dfrac{d}{dt} t &&\text{Por propiedades de la multiplicación }\\
\dfrac{d}{dt} (f\circ \gamma) &= t\dfrac{d}{dt} e^{2t^2}+e^{2t^2}\*\dfrac{d}{dt} t &&\text{Por propiedades de la multiplicación }\\
\dfrac{d}{dt} (f\circ \gamma) &= t2e^{2t^2}\dfrac{d}{dt}t^2 + e^{2t^2} &&\text{La derivada de la exponencial es ella misma por la derivada de su argumento}\\
\dfrac{d}{dt} (f\circ \gamma) &= 4t^2e^{2t^2} + e^{2t^2} &&\text{Derivando un monomio}\\
\dfrac{d}{dt} (f\circ \gamma) &= e^{2t^2}(4t^2 + 1) &&\text{Empleando factor común}
\end{align*}
por otra parte, por el primer caso de la regla de la cadena, obtenemos
\[ 	\dfrac{d}{dt}(f\circ g) = 	\dfrac{df}{dx}\dfrac{dx}{dt} + \dfrac{df}{dy}\dfrac{dy}{dt} \]
entonces calculemos las siguientes derivadas
\begin{align*}
\dfrac{\d f}{\d x}(xe^{x^2+y^2} ) &= e^{x^2+y^2}(1+2x^2)  &&\text{Aplicando la parcial a la función }\\
\dfrac{\d f}{\d y}(xe^{x^2+y^2}) &= 2xye^{x^2+y^2} &&\text{Aplicando la parcial a la función }\\
\dfrac{dx}{dt}(t) &= 1 &&\text{Derivando un termino lineal}\\
\dfrac{dy}{dt}(-t) &= -1 &&\text{Derivando un término lineal }\\
\end{align*}
Así, se tiene que 
\[ \dfrac{d}{dt}(f\circ g) = e^{x^2+y^2}(1+2x^2) -2xye^{x^2+y^2}  \] y como $ x = t  $ y $ y = -t $
\[ \therefore  \qquad \dfrac{d}{dt}(f\circ g) = e^{2t^2}(1+4t^2)  \]


\noindent2. Sea $f(u, v, w) = (e^{u -w}, \cos{(u + v)} + \sin{(u + v + w)})$ y $g(x,y) = (e^{x}, \cos{(y - x)}, e^{-y} )$. Calcule $ f\circ g$ y $\mathbf{D}(f\circ g)(0,0)$.\\

La composición está dada por 
\[ (f \circ g)(x,y) = f(g(x,y)) = f(e^x,cos(y-x),e^{-y}) \]
si aplicamos la regla de correspondencia de f, esto es:
\[ \left(  e^{e^x - e^{-y}},\cos(\cos(y-x) + e^x) + \sin(e^x + \cos(y-x+e^{-y}))  \right) \]
Empleando la regla de la cadena para el segundo caso tenemos que 
\[ D(f \circ g)(x,y) = D_f(g(x,y))D_g(x,y) \] ahora calculemos la derivada (matriz) de $ f $ como \[ D_f = \begin{pmatrix}[2.5]
\dfrac{\d f_1}{\d u} & \dfrac{\d f_1}{\d v} & \dfrac{\d f_1}{\d w} \\
\dfrac{\d f_2}{\d u} & \dfrac{\d f_2}{\d v} & \dfrac{\d f_2}{\d w} 
\end{pmatrix} \]
\begin{align*}
	Df(u,v,w) &= \begin{pmatrix}[2.5]
	\dfrac{\d f_1}{\d u} & \dfrac{\d f_1}{\d v} & \dfrac{\d f_1}{\d w} \\
	\dfrac{\d f_2}{\d u} & \dfrac{\d f_2}{\d v} & \dfrac{\d f_2}{\d w} 
	\end{pmatrix} \\
	Df(u,v,w) &= \begin{pmatrix}[2.5]
	\dfrac{\d }{\d u}e^{u -w} & \dfrac{\d }{\d v} e^{u -w}& \dfrac{\d }{\d w}e^{u -w} \\
	\dfrac{\d}{\d u}(\cos(u + v) + \sin(u + v + w)) & \dfrac{\d}{\d v} (\cos(u + v) + \sin(u + v + w)) & \dfrac{\d }{\d w} (\cos(u + v) + \sin(u + v + w))
	\end{pmatrix} \\
	Df(u,v,w) &= \begin{pmatrix}[2.5]
	e^{u -w} & 0& -e^{u -w} \\
	-\sin(u + v) + \cos(u + v + w) & -\sin(u + v) + \cos(u + v + w) & \cos(u + v + w)
	\end{pmatrix} \\
\end{align*}
Hagamos el mismo procedimiento para la función g
\begin{align*}
	Dg(x,y) &= \begin{pmatrix}[2.5]
	\dfrac{\d g_1}{\d x} & \dfrac{\d g_1}{\d y} \\
	\dfrac{\d g_2}{\d x} & \dfrac{\d g_2}{\d y} \\
	\dfrac{\d g_3}{\d x} & \dfrac{\d g_3}{\d y} 
	\end{pmatrix} &&\text{Planteando la jacobiana de la función }g \\
	Dg(x,y) &= \begin{pmatrix}[2.5]
	\dfrac{\d }{\d x} e^x & \dfrac{\d}{\d y} e^x \\
	\dfrac{\d }{\d x}\cos{(y - x)}  & \dfrac{\d}{\d y} \cos{(y - x)}\\
	\dfrac{\d }{\d x}e^{-y}  & \dfrac{\d}{\d y} e^{-y}
	\end{pmatrix} &&\text{Reemplazando los valores de }g_1,g_2,g_3 \\
	Dg(x,y) &= \begin{pmatrix}[1.5]
	e^x & 0\\
	\sin{(y - x)}  & -\sin{(y - x)}\\
	0 & -e^{-y}
	\end{pmatrix}&&\text{Aplicando las derivadas parciales } \\
\end{align*}

Ahora evaluemos $ Df(g(0,0)) =  $, para ello veamos que 
\begin{align*}
	g(0,0) &= (e^{0}, \cos{(0 - 0)}, e^{-0} ) && \text{Por la regla de correspondencia}\\
	g(0,0) &= (1, 1, 1 ) && \text{Evaluando dichos valores }
\end{align*}
entonces evaluaremos $ Df(1,1,1) $, esto es:
\begin{align*}
	Df(1,1,1) &= \begin{pmatrix}[2.5]
	e^{1 -1} & 0& -e^{1 -1} \\
	-\sin(1 + 1) + \cos(1 + 1 + 1) & -\sin(1 + 1) + \cos(1 + 1 + 1) & \cos(1 + 1 + 1)
	\end{pmatrix}\\
	Df(1,1,1) &= \begin{pmatrix}[2.5]
	1 & 0& 1 \\
	-\sin(2) + \cos(3) & -\sin(2) + \cos(3) & \cos(3)
	\end{pmatrix}\\
\end{align*}

también evaluemos $ Dg(0,0) $, \textit{i.e. }
\begin{align*}
	Dg(0,0)\begin{pmatrix}[2.5]
	e^0 & 0\\
	\sin{(0 - 0)}  & -\sin{(0 - 0)}\\
	0 & -e^{-0}
	\end{pmatrix}&&\text{Planteando la evaluación en la matriz de derivadas}\\
	Dg(0,0)\begin{pmatrix}
	1 & 0\\
	0 & 0\\
	0 & 1
	\end{pmatrix}&&\text{Dando valor a las entradas}\\
\end{align*}
Finalmente por el segundo caso de la regla de la cadena, sólo tenemos que multiplicar las matrices anteriores

\begin{align*}
	 D(f \circ g)(0,0) &= D_f(g(0,0))D_g(0,0) && \text{Por la regla de la cadena}\\
	 D(f \circ g)(0,0) &= \begin{pmatrix}[2.5]
	 1 & 0& 1 \\
	 -\sin(2) + \cos(3) & -\sin(2) + \cos(3) & \cos(3)
	 \end{pmatrix}\*\begin{pmatrix}
	 1 & 0\\
	 0 & 0\\
	 0 & 1
	 \end{pmatrix} && \text{Sustituyendo los valores correspondientes}\\
	 D(f \circ g)(0,0) &= \begin{pmatrix}[2.5]
	 1 + 0 + 0 & 0 + 0 + 1 \\
	 -\sin(2) + \cos(3) +0 +0 &  0 + 0 + -\cos(3)
	 \end{pmatrix} && \text{Efectuando multiplicación de matrices }\\
	 D(f \circ g)(0,0) &= \begin{pmatrix}[2.5]
	 1 & 1 \\
	 -\sin(2) + \cos(3)& -\cos(3)
	 \end{pmatrix} && \text{Eliminando los  }0\\
\end{align*}

\noindent3.  Calcule la derivada direccional de las siguientes funciones en el punto y la dirección dada: \\
\begin{theorem}[Derivada direccional]
	Si $ f:\R^2 \to \R $ es diferenciable, entonces todas las derivadas existen, además la derivada direccional en $ \vec{x} $ en la dirección de $ \vec{v} $ está dada por
	\[ Df(\vec{x}) = \nabla f(\vec{x}) \* \vec{v} = \left( \dfrac{\d}{\d x} f(\vec{x}) \right) v_1 +  \left( \dfrac{\d}{\d y} f(\vec{x}) \right) v_2 \]
	Donde $ \vec{v} = (v_1, v_2) $
\end{theorem}

a) $f(x,y) = x + 2xy -3y^2$, $(x_0, y_0) = (1,2)$ y $\vec{v} = \frac{3}{5}\hat{e}_1 + \frac{4}{5}\hat{e}_2 $.\\

\begin{align*}
	\nabla f(x,y) &= \left( \dfrac{\d}{\d x}  (x + 2xy -3y^2), \dfrac{\d}{\d y} (x + 2xy -3y^2) \right) && \text{Planteando el gradiente}\\
	\nabla f(x,y) &= \left( 1 + 2y , 2x -6y \right) && \text{Calculando las parciales }\\
	\\	
	\nabla f(1,2) \* \vec{v} &= \left( 1 + 2y , 2x -6y \right) \* \left(\dfrac{3}{5}, \dfrac{4}{5}\right) && \text{Por el teorema enunciado al inicio del ejercicio}\\
	\nabla f(1,2) \* \vec{v} &= \left( 1 + 2(2) , 2 -6(2) \right) \* \left(\dfrac{3}{5}, \dfrac{4}{5}\right) && \text{Sustituyendo por los valores de }x_0, y_0 \\
	\nabla f(1,2) \* \vec{v} &= \left( 5 ,-10 \right) \* \left(\dfrac{3}{5}, \dfrac{4}{5}\right) && \text{Evaluando los puntos}\\
	\nabla f(1,2) \* \vec{v} &= \left( 5\* \dfrac{3}{5} + (-10)\*\dfrac{4}{5} \right) && \text{Multiplicando entrada a entrada  }\\
	\nabla f(1,2) \* \vec{v} &= \left(3 + (-8)\right) && \text{Simplificando las fracciones  }\\
	\nabla f(1,2) \* \vec{v} &=-5 && \text{Operando}
\end{align*}
\newpage
b) $f(x,y) = \ln{\sqrt{x^2 + y^2}}$, $(x_0, y_0) = (1,0)$ y $\vec{v} = \left( \frac{1}{\sqrt{5}} \right)(2\hat{e}_1 + \hat{e}_2 )$.

\begin{align*}
	\nabla f(x,y) &= \left( \dfrac{\d}{\d x} \ln{\sqrt{x^2 + y^2}}, \dfrac{\d}{\d y}\ln{\sqrt{x^2 + y^2}} \right) && \text{Planteando el gradiente}\\
	\nabla f(x,y) &= \left( \dfrac{x}{x^2 + y^2}, \dfrac{y}{x^2 + y^2} \right) && \text{Calculando las derivadas parciales}\\
	\\	
	\nabla f(1,0) \* \vec{v} &= \left( \dfrac{(1)}{(1)^2 + (0)^2}, \dfrac{(0)}{(1)^2 + (0)^2}  \right) \* \left(\dfrac{2}{\sqrt{5}}, \dfrac{1}{\sqrt{5}} \right) && \text{Sustituyendo por los valores de } x_0, y_0\\
	\nabla f(1,0) \* \vec{v} &= \left(1,0\right) \* \left(\dfrac{2}{\sqrt{5}}, \dfrac{1}{\sqrt{5}} \right) && \text{Evaluando la expresión }\\
	\nabla f(1,0) \* \vec{v} &= \dfrac{2}{\sqrt{5}}&& \text{Multiplicando }\\
\end{align*}

c) $f(x,y) = e^x\cos{(\pi y)}$, $(x_0, y_0) = (0,-1)$ y $\vec{v} = - \left( \frac{1}{\sqrt{5}} \right)\hat{e}_1 +  \left( \frac{2}{\sqrt{5}} \right)\hat{e}_2 $.

\begin{align*}
	\nabla f(x,y) &= \left( \dfrac{\d}{\d x} (e^x\cos{(\pi y)}), \dfrac{\d}{\d y}(e^x\cos{(\pi y)}) \right) && \text{Planteando el gradiente}\\
	\nabla f(x,y) &= \left( e^x\cos(\pi y), -e^x\sin(\pi y) \right) && \text{Calculando las derivadas parciales}\\
	\\	
	\nabla f(0,-1) \* \vec{v} &= \left( e^0\cos(-\pi ), -e^0\sin(-\pi ) \right) \* \left(-\dfrac{1}{\sqrt{5}}, \dfrac{2}{\sqrt{5}} \right) && \text{Sustituyendo por los valores de } x_0, y_0\\
	\nabla f(0,-1) \* \vec{v} &= \left( \cos(-\pi ), -\sin(-\pi ) \right) \* \left(-\dfrac{1}{\sqrt{5}}, \dfrac{2}{\sqrt{5}} \right) && \text{Minimizando la expresión}\\
	\nabla f(0,-1) \* \vec{v} &= \left( \cos(\pi ), \sin(\pi ) \right) \* \left(-\dfrac{1}{\sqrt{5}}, \dfrac{2}{\sqrt{5}} \right) && \text{Porque seno es impar y coseno par }\\
	\nabla f(0,-1) \* \vec{v} &= \dfrac{2\sin(\pi)}{\sqrt{5}} - \dfrac{\cos(\pi)}{\sqrt{5}} && \text{Haciendo la multiplicación }\\
	\nabla f(0,-1) \* \vec{v} &= \dfrac{0}{\sqrt{5}} + \dfrac{1}{\sqrt{5}} && \text{Evaluando las trigonométricas }\\
	\nabla f(0,-1) \* \vec{v} &=  \dfrac{1}{\sqrt{5}} && \text{ }\\
\end{align*}

d) $f(x,y) = xy^2 + x^3y$, $(x_0, y_0) = (4,-2)$ y $\vec{v} =  \left( \frac{1}{\sqrt{10}} \right)\hat{e}_1 +  \left( \frac{3}{\sqrt{10}} \right)\hat{e}_2 $.
\begin{align*}
	\nabla f(x,y) &= \left( \dfrac{\d}{\d x} (xy^2 + x^3y), \dfrac{\d}{\d y}(xy^2 + x^3y) \right) && \text{Planteando el gradiente}\\
	\nabla f(x,y) &= \left( y^2 + 3x^2y ,2xy + x^3 \right) && \text{Calculando las derivadas parciales}\\
\end{align*}
\begin{align*}
	\nabla f(4,-2) \* \vec{v} &= \left(y^2 + 3x^2y ,2xy + x^3 \right) \* \left(\dfrac{1}{\sqrt{10}}, \dfrac{3}{\sqrt{10}} \right) && \text{Planteando la derivada direccional } \\
	\nabla f(4,-2) \* \vec{v} &= \left((-2)^2 + 3(4)^2(-2) ,2(4)(-2) + 4^3 \right) \* \left(\dfrac{1}{\sqrt{10}}, \dfrac{3}{\sqrt{10}} \right) && \text{Sustituyendo por los valores de } x_0, y_0\\
	\nabla f(4,-2) \* \vec{v} &= \left(-92,48 \right) \* \left(\dfrac{1}{\sqrt{10}}, \dfrac{3}{\sqrt{10}} \right) && \text{Efectuando el cálculo} \\
	\nabla f(4,-2) \* \vec{v} &= \dfrac{52}{\sqrt{10}} && \text{Multiplicando entrada a entrada} \\
\end{align*}


\noindent4. Encuentre un vector que sea normal a la curva $x^3 + xy + y^3 = 11$ en  $(1,2)$.\\

\begin{theorem}[El gradiente es  normal a la superficie de nivel]
	Sean $ f: U \subset \R^2 \to \R $ de clase $ \mathscr{C}^1 $ y $ (x_0, y_0) $ un punto sobre la superficie de nivel $\mathcal{S}  $ definida por $ f(x,y) = k, \quad k \in \R $. Entonces $ \nabla f(x_0, y_0) $ es normal a la superficie de nivel en el siguiente sentido. Si $ \vec{v} $ es el vector tangente en $ t = 0 $ de una trayectoria $ \vec{\gamma} $ en $ \mathcal{S} $ con $ \vec{\gamma}(0) = (x_0,y_0) $ entonces $ \nabla f(x_0,y_0) \* \vec{v} = 0 $
\end{theorem}

Para este ejercicio $ (x_0,y_0) = (1,2) $ y también $ f(x,y) = x^3+xy+y^3 $ con $ k = 11 $ por lo que sólo resta calcular el gradiente de la función 
\begin{align*}
	\nabla f(x,y) &= \left(\dfrac{\d}{\d x} (x^3+xy+y^3) , \dfrac{\d}{\d y} (x^3+xy+y^3)\right) && \text{Por definición del gradiente}\\
	\nabla f(x,y) &= \left(3x^2 + y , x+3y^2 \right) && \text{Calculando las parciales}\\
\end{align*}

Finalmente evaluamos en el punto deseado por lo que $ \nabla f(1,2) = (5,13) $ es el vector normal a la curva dada.\\

\noindent5.  El Capitán Ralphis se encuentra en problemas cerca del lado soleado de Mercurio. La temperatura del casco del barco cuando está en la ubicación $(x, y, z)$ estará dada por $T(x,y,z) = e^{-x^2 - 2y^2 - 3z^2}$, donde $x,y,z$ se miden en metros. Actualmente está en $(1,1,1)$.\\

a) ¿En qué direcciones debería proceder para disminuir la temperatura más rápidamente?\\

b) Si el barco viaja a $e^8$ metros por segundo, ¿qué tan rápido será la disminución de la temperatura si avanza en esa dirección?\\

c) Desafortunadamente, el metal del casco se romperá si se enfría a una velocidad superior a $\sqrt{14}e^2$ grados por segundo. Describa el conjunto de posibles direcciones en las que puede proceder a bajar la temperatura a no más de esa tasa.\\






\end{document}
