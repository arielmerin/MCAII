\documentclass[letterpaper,11pt]{article}

\usepackage[utf8]{inputenc}
\usepackage[spanish,mexico]{babel}
\usepackage{graphicx}
\usepackage{amsmath}
\usepackage{amsthm}
\usepackage{amsfonts}
%\usepackege{amssymb}
\usepackage[hmargin=1in, vmargin=1in]{geometry}
\usepackage{fancyhdr}
\pagestyle{fancy}
\usepackage{tasks}
\lhead{\ExiCarrera}
\chead{\ExiMateria}
\rhead{\ExiParcial}
\cfoot{\ExiEscuela}
\renewcommand{\headrulewidth}{0.4pt}
\renewcommand{\footrulewidth}{0.4pt}
\newtheorem{definition}{Definición}

\providecommand{\abs}[1]{\lvert#1\rvert}
\providecommand{\norm}[1]{\lVert#1\rVert}

%=================================================================================
%	Definición de comandos
%=================================================================================
\newcommand{\informacion}[1]{
\begin{center}
\fbox{\fbox{\parbox{\textwidth}{{\footnotesize#1}}}}
\end{center}
\vspace{5mm}}
\newcommand{\datos}{\makebox[0.7\textwidth]{Nombre:~\hrulefill} Fecha:~\hrulefill}
\newcommand{\pregunta}[2]{\item{#2}~{(#1 puntos)}\\ \vspace{5mm}						       
			{\bf Solución}}														  
%	Los siguientes comandos hay que definirlos desde aquí.								    %
\newcommand{\ExiCarrera}{Matemáticas para las Ciencias II.}											  %%%
\newcommand{\ExiMateria}{\textbf{Proyecto II}}														%%%%%%%%%%%%%%%%%%%%%%%%%%%%%
\newcommand{\ExiParcial}{Entrega: Viernes 20 de Marzo}														  %%%
\newcommand{\ExiEscuela}{\textbf{Facultad de Ciencias, UNAM}}	                		    %

%=================================================================================
%	Aquí comienza lo bueno, a definir el examen!!!!
%=================================================================================
\begin{document}



%%%%%%%%%%%%%%%%%%%%%            CARÁTULA            %%%%%%%%%%%%%%%%%%%%%%%%%
\setlength{\unitlength}{1cm}
\thispagestyle{empty}

\begin{center}
\textbf{\large Matemáticas para las Ciencias II}\\
\textbf{ Semestre 2020-1}\\
Prof. Pedro Porras Flores\\
Ayud. Irving Hérnandez Rosas \\
%Ayud. Carlos Rodolfo Barrera Anzaldot \\ [0.2cm]
\textbf{Proyecto I}
\end{center}
\rule{17cm}{0.3mm}

\noindent Realice los siguientes ejercicios, escribiendo el procedimiento claramente. Y recuerden que estos proyectos se entregan de manera individual en la plataforma de google classroom. 

\begin{definition}
	El conjutno de los puntos del plano tales que están a la misma distancia de una recta dada $ D $ y de un punto $ \vec{F} $, que no esté sobre $ D $, recibe el nombre de \textbf{parábola}
\end{definition}
\begin{enumerate}

% -----------------------------------------------------
% Problema uno
% -----------------------------------------------------

\item  De la definición de parábola deduzca de manera análoga como lo hicimos en la video-clase la ecuación para una parábola cuyo foco se encuentra en el eje $x$, es decir $$y^2 = 4px.$$ 


\begin{figure}[h!]
\centering
\caption{Parábola con foco sobre el eje $x$.}
\label{F1}
\end{figure}

% -----------------------------------------------------
% Problema dos
% -----------------------------------------------------
\newpage
\item De igual manera que se hizo en clase deduzca la ecuación de una elipse cuyos focos se encuentran sobre el eje $y$, esto es: $$\dfrac{x^2}{b^2} + \dfrac{y^2}{a^2}=1$$

\begin{figure}[h!]
\centering
\caption{Elipse con focos sobre el eje $y$.}
\label{F1}
\end{figure}

% -----------------------------------------------------
% Problema tres
% -----------------------------------------------------


\item Deduzca la ecuación de la hipérbola de la definición, sin importar donde estén los focos, es decir ya sea que muestre: $$\dfrac{x^2}{a^2} -  \dfrac{y^2}{b^2} = 1 \text{ o }  \dfrac{x^2}{b^2} - \dfrac{y^2}{a^2}=1$$

\begin{figure}[h!]
\centering
\caption{Hipérbola con focos sobre el eje $x$.}
\label{F1}
\end{figure}




\end{enumerate}




\end{document}
