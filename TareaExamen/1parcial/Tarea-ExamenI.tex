\documentclass[letterpaper,12pt]{article}
\usepackage[utf8]{inputenc}
\usepackage[spanish,mexico]{babel}
\usepackage{graphicx}
\usepackage{amsmath}
\usepackage{amsthm}
\usepackage{amsfonts}
\usepackage{subcaption}
\usepackage{mwe}
\usepackage[margin=1.5cm, vmargin={1.5cm,1.3cm},includefoot]{geometry}
\usepackage{fancyhdr}
\pagestyle{fancy}
\usepackage{tasks}
\lhead{\ExiCarrera}
\chead{\ExiMateria}
\rhead{\ExiParcial}
\cfoot{\ExiEscuela}
\renewcommand{\headrulewidth}{0.4pt}
\renewcommand{\footrulewidth}{0.4pt}

\providecommand{\abs}[1]{\lvert#1\rvert}
\providecommand{\norm}[1]{\lVert#1\rVert}														  
\newcommand{\ExiCarrera}{Matemáticas para las Ciencias II}											 
\newcommand{\ExiMateria}{\textbf{Tarea-examen I}}													
\newcommand{\ExiEscuela}{\textbf{Facultad de Ciencias, UNAM}}	                		   
\begin{document}

\setlength{\unitlength}{1cm}
\thispagestyle{empty}
\begin{picture}(19,3)
\put(-0.5,1.2){\includegraphics[scale=.20]{unam1.png}}
\put(16,1){\includegraphics[scale=.29]{fciencias1.png}}
\end{picture}

\begin{center}
\vspace{-114pt}
\textbf{\large Matemáticas para las Ciencias II}\\
\textbf{ Semestre 2020-2}\\
Prof. Pedro Porras Flores\\
Ayud. Irving Hernández Rosas \\
\textbf{Tarea-examen I}
\end{center}
\vspace{-20pt}
\rule{19cm}{0.3mm}

\noindent Realice los siguientes ejercicios, escribiendo el procedimiento claramente. Y recuerden que la tarea-examen se entrega individual. 

\noindent1. Muestre que $\mathbb{P}_{2}(\mathbb{R}) = \{ c  + bx + ax^2  \,  \vert \, a,b,c \in \mathbb{R} \}$, es un espacio vectorial con la suma usual y la multiplicación por escalar usual, es decir:
\begin{align*}
     + \colon & \mathbb{P}_{2}(\mathbb{R}) \times \mathbb{P}_{2}(\mathbb{R}) \longrightarrow \mathbb{P}_{2}(\mathbb{R}) \\
    & (a_1 x^2 + b_1x + c_1 , a_2 x^2 + b_2x + c_2) \mapsto  (a_1 + a_2)x^2 + (b_1 + b_2)x + (c_1 + c_2). \\
    \mu \colon & \mathbb{R} \times \mathbb{P}_{2}(\mathbb{R}) \longrightarrow \mathbb{P}_{2}(\mathbb{R}) \\
    & (\alpha, (a_1 x^2 + b_1x + c_1)) \mapsto  (\alpha a_1)x^2 + (\alpha b_1)x + (\alpha c_1).
 \end{align*}


\noindent2. Muestre que el conjunto  $\beta =  \{ 1, x , x^2 \}$ es base de $ \mathbb{P}_{2}(\mathbb{R})$

\noindent3. Muestre que la siguiente transformación es lineal.

\begin{align*}
     T \colon & \mathbb{P}_{2}(\mathbb{R})  \longrightarrow \mathbb{P}_{2}(\mathbb{R}) \\
     & T(f(x)) \mapsto  xf^{'}(x) +x f(2) + f(3). \\
    %& T(f(x)) \mapsto  xf^{'}(x) +\dfrac{1}{2} f^{''}(x). \\
 \end{align*}

\noindent4. Determine el núcleo y la imagen de $T$.

\noindent5. Encuentre la matriz asociada a $T$ con respecto a la base $\beta$, esto es $[T]_{\beta}$.

\noindent6. ¿Cuál es el rango de $[T]_{\beta}$?

\noindent7. La matriz $[T]_{\beta}$ es invertible, si sí muéstrelo, si no argumente porque.

\noindent8. ¿Cuales son los valores propios asociados a $[T]_{\beta}$?

\noindent9. Determine los vectores propios asociados a cada valor propio. 

\noindent10. Muestre que el conjunto de los vectores propios es una base ordenada.

\noindent11. Determine $Q \in M_{3\times 3}(\mathbb{R})$, tal que $Q^{-1}[T]_{\beta} Q = D$, donde $D$ es una matriz diagonal cuyos elementos de la diagonal son valores propios.

\noindent12. Muestre que $\beta^{'} =\{ -3+x , -3-13x + 4x^2, 1+x \}$, es una base para $ \mathbb{P}_{2}(\mathbb{R})$ y además determine $[T]_{\beta^{'}}$ .


\end{document}
