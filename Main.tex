\documentclass[letterpaper]{article}
\usepackage[utf8]{inputenc}
\usepackage[spanish, es-nodecimaldot]{babel}
\usepackage{amssymb, amsmath}
\usepackage{graphicx}
\usepackage{lipsum}
\usepackage{dsfont}
\usepackage[margin=2cm,
vmargin={1.5cm,1.3cm},
includefoot]{geometry}
\usepackage{setspace}
\usepackage{subcaption}
\usepackage{tocloft}
\usepackage{upgreek}
\usepackage{amsthm}
\usepackage{graphicx}
\usepackage{paralist}

\newcommand{\V}{\mathds{V}}

\newcommand{\F}{\mathds{F}}

\newcommand{\tq}{ \quad \cdot  \backepsilon \cdot \quad }

\newcommand{\ld}{\lim\limits_{x \to 0^{+}}}

\newcommand{\li}{\lim\limits_{x \to 0^{-}}}

\newcommand{\la}{\lim\limits_{x \to a}}

\newcommand{\R}{\mathds{R}}

\renewcommand{\*}{\cdot}

\newtheorem{theorem}{Teorema}[section]
\theoremstyle{definition}
\newtheorem{definition}{Definición}[section]

\begin{document}
		\begin{titlepage}
		\begin{center}
			\Large{\textbf{Espacios vectoriales}} \\[0.1cm]
			\huge{Matemáticas para las ciencias aplicadas II}\\[0.2cm]
			\large{Aquino Chapa Armando Abraham y Merino Peña Kevin Ariel }
			\\
			\today
		\end{center}
		\let\newpage\relax% Avoid following page break
		\hrulefill
	\end{titlepage}
	\section{Definición de espacio vectorial}

	
	1. Escribe el vector cero en $M_{3x4}(\mathbb{R})$
		\begin{definition}[Matriz]
		Una \textbf{Matriz} es un arreglo rectangular de elementos de un campo $ \F(\R) $ de la forma
		\begin{equation*}
		A_{m,n} = 
		\begin{pmatrix}
		a_{1,1} & a_{1,2} & \cdots & a_{1,n} \\
		a_{2,1} & a_{2,2} & \cdots & a_{2,n} \\
		\vdots  & \vdots  & \ddots & \vdots  \\
		a_{m,1} & a_{m,2} & \cdots & a_{m,n} 
		\end{pmatrix}
		\end{equation*}
		A los elementos $ a_{i,j} $ con $ 1 \leq j \leq n $ y $ 1 \leq i \leq m $ se les llama entradas de la matriz, a las matrices las denotamos por $ \mathds{A} $ 				\textit{(letras mayúsculas)} y al conjunto de las matrices de mn se les denota por $ M_{m\times n}(\F) $
		
	\end{definition}
	De esta manera tenemos que el vector cero de la matriz de 3 renglones por 4 columnas es aquella cuyas entradas (todas) son 0 \textit{i. e. }
		\begin{equation*}
		A_{3x4} = 
		\begin{pmatrix}
		a_{1,1} & a_{1,2} & a_{1,3} & a_{1,4} \\
		a_{2,1} & a_{2,2} & a_{2,3} & a_{2,4} \\
		a_{3,1} & a_{3,2} & a_{3,3} & a_{3,4} 
		\end{pmatrix}
		= 
		\begin{pmatrix}
		0 & 0 & 0 & 0\\
		0 & 0 & 0 & 0 \\
		0 & 0 & 0 & 0
		\end{pmatrix}
		\end{equation*}
		
 2. Sea $V$ el conjunto de todas las funciones diferenciables definidas en $\mathbb{R}$. Muestre que $V$ es un espacio vectorial con las operaciones usuales de suma y multiplicación por un escalar para funciones. 
	
	Veamos que la derivada cumple las siguientes propiedades
	\begin{align*}
	(f(x) + g(x))' &= \lim_{h\to 0}
	\end{align*}
\end{document}
