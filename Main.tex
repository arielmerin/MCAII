\documentclass[letterpaper]{article}
\usepackage[utf8]{inputenc}
\usepackage[spanish, es-nodecimaldot]{babel}
\usepackage{amssymb, amsmath}
\usepackage{graphicx}
\usepackage{lipsum}
\usepackage{dsfont}
\usepackage[margin=2cm,
vmargin={1.5cm,1.3cm},
includefoot]{geometry}
\usepackage{setspace}
\usepackage{subcaption}
\usepackage{tocloft}
\usepackage{upgreek}
\usepackage{amsthm}
\usepackage{graphicx}
\usepackage{paralist}

\newcommand{\V}{\mathds{V}}

\newcommand{\F}{\mathds{F}}

\newcommand{\tq}{ \quad \cdot  \backepsilon \cdot \quad }

\newcommand{\ld}{\lim\limits_{x \to 0^{+}}}

\newcommand{\li}{\lim\limits_{x \to 0^{-}}}

\newcommand{\la}{\lim\limits_{x \to a}}

\newcommand{\R}{\mathds{R}}

\renewcommand{\*}{\cdot}

\newtheorem{theorem}{Teorema}[section]
\theoremstyle{definition}
\newtheorem{definition}{Definición}[section]

\begin{document}
		\begin{titlepage}
		\begin{center}
			\Large{\textbf{Espacios vectoriales}} \\[0.1cm]
			\huge{Matemáticas para las ciencias aplicadas II}\\[0.2cm]
			\large{Aquino Chapa Armando Abraham y Merino Peña Kevin Ariel }
			\\
			\today
		\end{center}
		\let\newpage\relax% Avoid following page break
		\hrulefill
	\end{titlepage}
	\section{Definición de espacio vectorial}
	\begin{definition}
		Definimos $ \R^2 $ como el conjunto de pares ordenados $ (x,y) $ tal que $ x $ es un real y $ y $ es un real.
		\[ \R^2 = \{(x,y) | x \in \R \land y \in \R \} \]
	\end{definition}
	\begin{definition}
		\textbf{Campo}, conjunto no vacío con operaciones binarias (+,*) 
		$ \tq $ cumplen las siguientes 11 propiedades
		\begin{enumerate}[(I)]
			\item Cerrado bajo la suma
			\[x,y \in \R \implies x + y \in \R \]
			\item  Cerrado bajo el producto
			\[  x,y \in \R \implies x\cdot y \in \R \]
			\item Conmutatividad para la suma
			\[  x,y \in \R \implies x + y = y + x\]
			\item Conmutatividad para el producto
			\[  x,y \in \R \implies x \cdot y  = y\cdot x \]
			\item Asociatividad para el producto
			\[  x,y,z \in \R \implies x\* y = y \* x \]
			\item Asociatividad para la suma
			\[  x,y,z  \in \R \implies x +(y+z) = y + (x +z)  \]
			\item Neutro aditivo
			\[ \exists 0 \in \R \tq x + 0 = x \]
			\item Neutro multiplicativo
			\[ \exists 1 \in \R \tq x \* 1 = x  \]
			\item Inverso aditivo
			\[ \exists n  \in \R \tq x + n = 0  \]
			\item Inverso multiplicativo
			\[ \exists n \in \R \backslash \{0\} \tq x\* n = 1 \]
			\item Distributividad
			\[ x,y,z \in \R \tq x \*(y+z) = x\* y + x \* z  \]
		\end{enumerate}
	\end{definition}
	\begin{definition}
		\textbf{Espacio vectorial} Sea $ \V $ un conjunto  no vacío, con dos operaciones binarias y $ \F $ un campo, diremos que $ \V $ es espacio vectorial si cumple
		\begin{enumerate}[(i)]
			\item Conmutatividad para la suma
			\[ \vec{x}, \vec{y} \in \V \implies \vec{x} + \vec{y} = \vec{y} + \vec{x} \]
			\item Asociatividad para la suma
			\[ \vec{x},\vec{y},\vec{z} \in \V \implies (\vec{x} + \vec{y}) + \vec{z} = ( \vec{z} + \vec{y}) + \vec{x}  \]
			\item Neutro Aditivo
			\[ \exists \vec{0}\in \V \tq \vec{x} + \vec{0} = \vec{x} \]
			\item Inverso aditivo 
			\[\forall \vec{x} \in \V \quad \exists \vec{y} \in \V \tq \vec{x} + \vec{y} = 0 \]
			\item Neutro multiplicativo
			\[\forall \vec{x} \in \V \implies 1\* \vec{x} = \vec{x} \]
			\item Asociatividad para la multiplicación
			\[ \forall \ \alpha, \beta \in \F \ \land \ \vec{x} \in \V \implies (\alpha \beta)\vec{x} = \alpha(\beta \vec{x}) \]
			\item Distributividad entre escalares y vectores
			\[  \forall \ \alpha, \beta \in \F \ \land \ \vec{x} \in \V \implies (\alpha + \beta)\vec{x} = \alpha \vec{x} + \beta \vec{x} \]
			\item Distributividad entre vectores y escalares
			\[  \forall \ \alpha \in \F \ \land \ \vec{x} \in \V \implies \alpha (\vec{x} + \vec{y}) = \alpha \vec{x} + \alpha \vec{y} \]	
		\end{enumerate}
	\end{definition}
	\subsection{Definición de las operaciones binarias en espacios vectoriales}
\newpage
	
	1. Escribe el vector cero en $M_{3x4}(\mathbb{R})$
		\begin{definition}[Matriz]
		Una \textbf{Matriz} es un arreglo rectangular de elementos de un campo $ \F(\R) $ de la forma
		\begin{equation*}
		A_{m,n} = 
		\begin{pmatrix}
		a_{1,1} & a_{1,2} & \cdots & a_{1,n} \\
		a_{2,1} & a_{2,2} & \cdots & a_{2,n} \\
		\vdots  & \vdots  & \ddots & \vdots  \\
		a_{m,1} & a_{m,2} & \cdots & a_{m,n} 
		\end{pmatrix}
		\end{equation*}
		A los elementos $ a_{i,j} $ con $ 1 \leq j \leq n $ y $ 1 \leq i \leq m $ se les llama entradas de la matriz, a las matrices las denotamos por $ \mathds{A} $ 				\textit{(letras mayúsculas)} y al conjunto de las matrices de mn se les denota por $ M_{m\times n}(\F) $
		
	\end{definition}
	De esta manera tenemos que el vector cero de la matriz de 3 renglones por 4 columnas es aquella cuyas entradas (todas) son 0 \textit{i. e. }
		\begin{equation*}
		A_{3x4} = 
		\begin{pmatrix}
		a_{1,1} & a_{1,2} & a_{1,3} & a_{1,4} \\
		a_{2,1} & a_{2,2} & a_{2,3} & a_{2,4} \\
		a_{3,1} & a_{3,2} & a_{3,3} & a_{3,4} 
		\end{pmatrix}
		= 
		\begin{pmatrix}
		0 & 0 & 0 & 0\\
		0 & 0 & 0 & 0 \\
		0 & 0 & 0 & 0
		\end{pmatrix}
		\end{equation*}
		
	
	
\end{document}
