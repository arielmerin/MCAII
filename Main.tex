\documentclass[letterpaper]{article}
\usepackage[utf8]{inputenc}
\usepackage[spanish]{babel}
\usepackage{amssymb, amsmath}
\usepackage{graphicx}
\usepackage{lipsum}
\usepackage{dsfont}
\usepackage[margin=1.3cm,
vmargin={1.3cm,1.3cm},
includefoot]{geometry}
\usepackage{setspace}
\usepackage{subcaption}
\usepackage{tocloft}
\usepackage{upgreek}
\usepackage{amsthm}
\usepackage{graphicx}
\usepackage{paralist}
\usepackage{tasks}

\newcommand{\V}{\mathds{V}}

\newcommand{\F}{\mathds{F}}

\newcommand{\tq}{ \quad \cdot  \backepsilon \cdot \quad }

\newcommand{\ld}{\lim\limits_{x \to 0^{+}}}

\newcommand{\li}{\lim\limits_{x \to 0^{-}}}

\newcommand{\la}{\lim\limits_{x \to a}}

\newcommand{\R}{\mathds{R}}

\renewcommand{\*}{\cdot}

\makeatletter
\renewcommand*\env@matrix[1][*\c@MaxMatrixCols c]{%
	\hskip -\arraycolsep
	\let\@ifnextchar\new@ifnextchar
	\array{#1}}
\makeatother

\newtheorem{theorem}{Teorema}[section]
\theoremstyle{definition}
\newtheorem{definition}{Definición}

\begin{document}
		\begin{titlepage}
		\begin{center}
			\Large{\textbf{Espacios vectoriales}} \\[0.1cm]
			\huge{Matemáticas para las ciencias aplicadas II}\\[0.2cm]
			\large{Aquino Chapa Armando Abraham y Merino Peña Kevin Ariel }
			\\
			\today
		\end{center}
		\let\newpage\relax% Avoid following page break
		\hrulefill
	\end{titlepage}

	
	\noindent \textbf{1}. Escribe el vector cero en $M_{3x4}(\mathbb{R})$
		\begin{definition}[Matriz]
		Una \textbf{Matriz} es un arreglo rectangular de elementos de un campo $ \F(\R) $ de la forma
		\begin{equation*}
		A_{m,n} = 
		\begin{pmatrix}
		a_{1,1} & a_{1,2} & \cdots & a_{1,n} \\
		a_{2,1} & a_{2,2} & \cdots & a_{2,n} \\
		\vdots  & \vdots  & \ddots & \vdots  \\
		a_{m,1} & a_{m,2} & \cdots & a_{m,n} 
		\end{pmatrix}
		\end{equation*}
		A los elementos $ a_{i,j} $ con $ 1 \leq j \leq n $ y $ 1 \leq i \leq m $ se les llama entradas de la matriz, a las matrices las denotamos por $ \mathds{A} $ 				\textit{(letras mayúsculas)} y al conjunto de las matrices de mn se les denota por $ M_{m\times n}(\F) $
		
	\end{definition}
	De esta manera tenemos que el vector cero de la matriz de 3 renglones por 4 columnas es aquella cuyas entradas (todas) son 0 \textit{i. e. }
		\begin{equation*}
		A_{3x4} = 
		\begin{pmatrix}
		a_{1,1} & a_{1,2} & a_{1,3} & a_{1,4} \\
		a_{2,1} & a_{2,2} & a_{2,3} & a_{2,4} \\
		a_{3,1} & a_{3,2} & a_{3,3} & a_{3,4} 
		\end{pmatrix}
		= 
		\begin{pmatrix}
		0 & 0 & 0 & 0\\
		0 & 0 & 0 & 0 \\
		0 & 0 & 0 & 0
		\end{pmatrix}
		\end{equation*}
		
 \noindent \textbf{2}. Sea $V$ el conjunto de todas las funciones diferenciables definidas en $\mathbb{R}$. Muestre que $V$ es un espacio vectorial con las operaciones usuales de suma y multiplicación por un escalar para funciones. 
	
	Veamos que la derivada cumple las siguientes propiedades
	\begin{align*}
	(f(x) + g(x))' &= \lim_{h\to 0} \dfrac{f(x +h) + g(x + h) - (f(x) + g(x))}{h}\\
	& = \lim_{h\to 0} \dfrac{f(x+h) - f(x) + g(x+h) - g(x)}{h}\\
	&= \lim_{h\to 0} \dfrac{f(x+h)-f(x)}{h} + \lim_{h\to 0} \dfrac{g(x+h)-g(x)}{h}\\
	&=f'(x) + g'(x)
	\end{align*}
	Así hemos probado que la derivada abre sumas
	\begin{align*}
	(cf(x))' &= \lim_{h\to 0} \dfrac{cf(x + h) - cf(x)}{h}\\
	& = c\lim_{h\to 0} \dfrac{f(x+h)-f(x)}{h}\\
	&= cf'(x)
	\end{align*}
	De esta manera queda conolidado que en la función derivada, los escalares son sacados de la función
	
	\begin{align*}
	f'(x) &= \lim_{h\to 0} \dfrac{f(x + h) - f(x)}{h}\\
	& = \lim_{h\to 0} \dfrac{c-c}{h}\\
	&= 0
	\end{align*}
	Esto se vale para cualquier constante, en particular el 0\\
	
 \noindent \textbf{3}. Prueba que el conjunto de las funciones pares en $\mathbb{R}$ es un espacio vectorial con suma y multiplicación por escalar usuales para funciones. Recuerde que una función es par si $\forall x \in Dom(f)$ entonces $f(-x) = f(x)$ \\
 	
 	Si tenemos en cuenta que $f(-t) + g(-t) = f(t) + g(t)$ y que si tenemos constantes siempre ocurre que $cf(-t) = cf(t) $ entonces ya hemos probado las dos primeras condiciones y para hallar el neutro basta con usar el 0 del cambo $(\R)$ para notar que también lo manda al 0 vector.\\
 	
\noindent \textbf{4}. Sea $V$ el conjunto de pares ordenados de números reales. Si $(a_{1},a_{2})$ y $(b_{1},b_{2})$ son elementos de $V$ y $\alpha \in \mathbb{R}$, definamos la suma y multiplicación escalar de la siguiente manera:
\begin{enumerate}
\item[(i)] $(a_{1},a_{2}) + (b_{1},b_{2}) = (a_{1} + b_{1}, a_{2}b_{2})$ 
\item[(ii)] $\alpha(a_{1},a_{2}) = (\alpha a_{1},a_{2})$.\\
\end{enumerate}
¿Es $V$ un espacio vectorial sobre $\mathbb{R}$ con estas operaciones?\\

No puede ser un espacio vectorial porque si tenemos que \[ 0(a_1, a_2) = (0, a_2)  \] para cumplir el cero vector, entonces se compliría para cualquier $a_2$ lo cual no es posible pues contradice la unicidad del cero.\\

\noindent 5. Determinar cuales de los siguientes conjuntos son subespacios de $\mathbb{R}^{3}$ bajo las operaciones de suma y multiplicación por un escalar usual.\\
\begin{definition}
	Sea $ \mathcal{U} $ un subconjunto de $ \mathcal{V}$ espacio vectorial sobre $ \F $ decimos que $ \mathcal{U} $ es un subespacio vectorial de $ \mathcal{V} $ si cumple lo siguiente
	
	\begin{enumerate}[i)]
		\item $ \vec{0}  \in \mathcal{U} $
		\item $ \forall \vec{u}, \vec{v} \in \mathcal{U} \implies \vec{u} + \vec{v} \in \mathcal{U} $
		\item Sea $ \alpha \in \F, \vec{u} \in \mathcal{U} \implies \alpha \* \vec{u} \in \mathcal{U} $
	\end{enumerate}
	   
\end{definition}

\begin{tasks}(1)
\task $W_{1} = \lbrace (a_{1},a_{2},a_{3}) \in \mathbb{R}^{3} \big\vert  a_{1}=3a_{2}$ y $a_{3}=-a_{2} \rbrace$

Veamos que $ W_{1} $ contiene a $ \vec{0} $ esto es que algún elemento en $ W_{1} = (0,0,0) $ por lo que 
	\begin{align*}
		(0,0,0) &= (a_{1},a_{2},a_{3}) && \text{Por $ \vec{0} \in \R^3 $}\\
		&= (3a_{2},a_{2},-a_{2}) && \text{Por $a_{1}=3a_{2}$ y $a_{3}=-a_{2}$}\\
		&= (3(0),(0),-(0)) && \text{Para cualquier $ a_2 $}\\
		&= (0,0,0)
	\end{align*}
	
Por otra parte comprobemos que la suma está dentro de $ W_1 $ 

Sean $ \hat{u} = (a_{1},a_{2},a_{3})  $ y $\hat{v} (b_{1},b_{2},b_{3})  \in W_1$ la suma de vectores se realiza entrada a entrada por lo que
\begin{align*}
	(a_{1},a_{2},a_{3}) + (b_{1},b_{2},b_{3}) &= (3a_{2},a_{2},-a_{2}) + (3b_{2},b_{2},-b_{2})\\
	&= (3 a_2 + 3b_2, a_2 + b_2, -a_2 - b_2)\\
	&= (3( a_2 + b_2), (a_2 + b_2), -(a_2 +b_2))
\end{align*}
 Y como $ a_1 + b_2 \in \R^3$ entonces $ \hat{u} + \hat{v} \in W_1 $ por lo que cumple $ II) $\\
 
 Finalmente veamos que si $ k \in R, \hat{u} \in W_1 \implies k\hat{u} \in W_1$
 
	\begin{align*}
		k(3a_2, a_2, -a_2) \in W_1\\
		(3ka_2, ka_2, -ka_2) \in W_1
	\end{align*}
	
	Por lo tanto cumple $ III) $ 
	
	$ \therefore W_1  $ es subespacio vectorial de $ \R^3 $
\task $W_{2} = \lbrace (a_{1},a_{2},a_{3}) \in \mathbb{R}^{3} \big\vert  a_{1} = a_{3} + 2 \rbrace$

Veamos si $ \hat{0} \in W_2 $ si esto ocurriera entonces $ (0,0,0)  \in W_2$ l que significaría lo siguiente
\begin{align*}
	(0,0,0) &= (a_3 + 2, a_2, a_3) && \text{Por que deben ser iguales entrada a entrada}\\
	0 &= a_3 + 2\\
	0 &= a_2 \\
	0 &= a_3 
\end{align*}
Podemos observar que en esta situación, $ a_3 = -2 \land a_3 = 0 $ lo cual no es posible, dicha contradicción vino de suponer que $ \hat{0} \in W_2 $

\[ \therefore \hat{0} \notin W_2  \] por lo que $ W_2 $ \textbf{no} es subespacio vectorial de $ \R^3 $


\task $W_{3} = \lbrace (a_{1},a_{2},a_{3}) \in \mathbb{R}^{3} \big\vert  2a_{1} - 7a_{2} + a_{3} = 0 \rbrace$

Notemos que en la declaración de los elementos de $ W_3 $ podemos deducir que \[ a_3 = 7a_2 - 2 a_1 \] entonces $ \hat{u} \in W_3 \implies \hat{u} = (a_1, a_2, 7a_2 - 2a_1)  $

Veamos que para cumplir $ I) $ el vector cero debería estar en $ W_1 $ \textit{i.e.} 

\begin{align*}
	(0,0,0) &= (a_1, a_2, 7a_2 - 2 a_1) \\
	0 &=  a_1\\
	0 &=  a_2\\
	0 &=  7a_2 -2 a_1
\end{align*}
 lo anterior se cumple si $ a_2 = 0 = a_1 $
 
 $ \therefore \hat{0} \in W_3 $
 
 Ahora, sean $ \hat{u}, \hat{v} \in W_3 \implies \hat{u} = (a_1, a_2, 7a_2 - 2a_1) \land \hat{v} = (b_1, b_2, 7b_1 - 2 b_1) $ y probemos que $ \hat{u} + \hat{v} \in W_3 $
 
 \begin{align*}
 	(a_1, a_2, 7a_2 - 2a_1) + (b_1, b_2, 7b_1 - 2 b_1) &= (a_1 + b_1, a_2 + b _2, 7a_2 + 7b_2 -2a_1 - 2b_1)\\
 	&= (a_1 + b_1, a_2 + b _2, 7(a_2 + b_2) -2(a_1 +b_1))
 \end{align*}
 y como $ a_1 + b_1 \in R $ también se encontrarán dentro de $ W_3 $ por lo que la suma es cerrada en el conjunto $ W_3 $
 
 Por último veamos que si $ k \in \R, \hat{u} \in W_3 \implies k\* \hat{u} \in W_3 $
	\begin{align*}
		k\hat{u} &= k(a_1, a_2, 7a_2 - 2a_1)\\
		k\hat{u} &= (ka_1, ka_2, 7ka_2 - 2ka_1)
	\end{align*}
	De lo anterior podemos concluir que cada uno de esos $ ka_1, ka_2 $  elementos estarán en $ \R  $ por lo que $ k\hat{u} $ resultarán también estar en $ W_3 $
	
	$ \therefore W_3$ es un subespacio vectorial de $ \R^3 $

\task $W_{4} = \lbrace (a_{1},a_{2},a_{3}) \in \mathbb{R}^{3} \big\vert  a_{1} - 4a_{2} - a_{3} = 0 \rbrace$
De la definición de los elementos de $ W_4 $ se sigue que si $ \hat{u} $ es un elemento de este conjunto, tendrá la forma $ \hat{u}= (4a_2 + a_3, a_2, a_3) $
Comencemos averiguando si $ W_4 $ tiene elemento neutro, \textit{i. e.}

\begin{align*}
	(0,0,0) &= (4a_2 + a_3, a_2, a_3) \\
	0 &= 4a_2 + a_3 && \text{para ser iguales entrada a entrada}\\
	0 &= a_2 && \\
	0 &= a_3 && 
\end{align*}
Lo anterior ocurre cuando $ a_2 = a_3 = 0 $ por lo que $ \hat{0} \in W_4 $ y así cumple la condición \textit{I)}

Siguiendo con la comprobación de sus propiedades como subespacio vectorial, tenemos que: Sean $ \hat{u}, \hat{v} \in W_4 \implies \hat{u} + \hat{v} \in W_4 $ \textit{i. e.}

\begin{align*}
		\hat{u} + \hat{v} &= (4a_2 + a_3, a_2, a_3) + (4b_2 + b_3, b_2, b_3)\\
		&= (4a_2 + 4b_2 + b_3 + a_3, a_2 + b_2, a_3 + b_3) && \text{sumando entrada por entrada}\\
		&= (4(a_2 + b_2) + (b_3 + a_3), (a_2 + b_2), (a_3 + b_3)) && \text{asociatividad y distributividad en $ \R $}\\
\end{align*}
y como $ (a_2 + b_2) \in \R $ la suma de $ \hat{u}, \hat{v} \in W_4 $

Finalmente notemos que si $ k \in \R, \hat{u} \in W_4 \implies k\* \hat{u} \in W_4 $ 

\begin{align*}
	k\hat{u} & = k(4a_2 + a_3, a_2, a_3)\\
	 & = (4ka_2 + ka_3, ka_2, ka_3) && \text{por distributividad}
\end{align*}

y $ ka_2, k_3 \in \R $ entonces $ k\*\hat{u} \in W_4 $

$ \therefore W_4 $ es subespacio vectorial de $ \R^3 $
\end{tasks}


\noindent \textbf{6}. En cada caso diga si los vectores son generados por el conjunto $S$ \\

\begin{definition}
	Sea $ \mathcal{S} $ un subconjunto de un espacio vectorial $ \mathcal{V} $ decimos que $ \mathcal{S} $ genera a $ \mathcal{V} $ si $ \forall \hat{x} \in \mathcal{V} $ es una combinación lineal de elementos de $ \mathcal{S} $ al generado de s se le denota como $ span(\mathcal{S}), <\mathcal{S}>, gen(\mathcal{S})$ 
\end{definition}

\textbf{a)} $(2,-1,1), S =  \lbrace (1,0,2),(-1,1,1) \rbrace$

Sea $\alpha _1$, $\alpha _2 \in \mathbb{R}$.

Entonces $(2,-1,1) = \alpha _{1}(1,0,2) + \alpha _{2}(-1,1,1) = (\alpha _{1}, 0, 2\alpha _{1})+ (-\alpha _{2},\alpha _{2},\alpha _{2}) = \alpha _{1}-\alpha _{2},\alpha _{2},2\alpha _{1}+\alpha _{2}$.

Tenemos el siguiente sistema de ecuaciones:
\begin{eqnarray*}
\alpha_{1}-\alpha_{2}= 2 \\
\alpha_{2} = -1\\
2\alpha_{1}+\alpha_{2}=1\\
\end{eqnarray*}
Ahora:
\begin{eqnarray*}
\alpha_{1} -(-1)=2 \\
\alpha_{2}=-1\\
2\alpha_{1} + \alpha_{2} = 1\\
\end{eqnarray*}
Al resolver el sistema, obtenemos:
\begin{eqnarray*}
\alpha_{1}=1 \\
\alpha_{2} = -1 \\
1 = 1
\end{eqnarray*}
Entonces:\\ $1(1,0,2) + (-1)(-1,1,1) = (1,0,2) + (1,-1,-1) = (2,-1,1)$\\
Cómo el sistema de ecuaciones si se satisface, el conjunto $S$ SI genera al vector $(2,-1,-1)$\\ \\

\noindent \textbf{b)} $(2,-1,1,3), S =  \lbrace (1,0,1,-1),(0,1,1,1) \rbrace$

Sea $\alpha _1$, $\alpha _2 \in \mathbb{R}$.\\
Entonces: $(2,-1,1,3)=\alpha_{1}(1,0,1,-1)+\alpha_{2}(0,1,1,1)= (\alpha_{1},0,\alpha_{1},-\alpha_{1})+(0,\alpha_{2},\alpha_{2},\alpha_{2})= \alpha_{1},\alpha_{2},\alpha_{1}+\alpha_{2},-\alpha_{1}+\alpha_{2}$.\\
Tenemos el siguiente sistema de ecuaciones:\\
\begin{eqnarray*}
\alpha_{1}=2\\
\alpha_{2} = -1\\
\alpha_{1}+\alpha_{2}=1\\
-\alpha_{1}+\alpha_{2}=3\\
\end{eqnarray*}
Ahora:\\
\begin{eqnarray*}
\alpha_{1}=2\\
\alpha_{2}= -1\\
2-1  =1\\
-(-1)+2 = 3\\
\end{eqnarray*}
Por último:
\begin{eqnarray*}
\alpha_{1}=2\\
\alpha_{2}=-1\\
1=1\\
3=3\\
\end{eqnarray*}
Al resolver el sistema de ecuaciones verificamos si el conjunto $S$ genera al vector. Entonces:
\[2(1,0,1,-1)+(-1)(0,1,1,1)=(2,0,2,-2)+(0,-1,-1,-1)=(2,-1,-1,-3)\]
Como el producto de los escalares por los elementos del conjunto $S$ no forman al vector, podemos concluir que $S$ \textbf{NO} genera a $(2,-1,1,3)$.\\

\textbf{c)} $2x^3 - x^2 + x + 3, S =  \lbrace x^3 + x^2 + x +1, x^2 + x +1, x +1 \rbrace$\\

Sean $\alpha_1 $, $\alpha_2$ y $\alpha_3$ elementos del campo, si suponemos que $2x^3 - x^2 + x + 3$ es generado por $S$ implicará que existen dichos 3 elementos $\tq$ 
\[ 2x^3 - x^2 + x + 3 = \alpha_1 (x^3 + x^2 + x +1) + \alpha_2 (x^2 + x +1) + \alpha_3 (x +1) \]

\begin{eqnarray*}
	\alpha_1 x^3 + \alpha_1 x^2 + \alpha_1 x + \alpha_1 	\\
	\alpha_2 x^2 + \alpha_2 x + \alpha_2 \\
	\alpha_3 x + \alpha_3 
\end{eqnarray*}
Por lo que ocurre lo siguiente
\[ 2x^3 - x^2 + x + 3 = \alpha_1 x^3 + \alpha_1 x^2 + \alpha_1 x + \alpha_1 + \alpha_2 x^2 + \alpha_2 x + \alpha_2 +\alpha_3 x + \alpha_3  \]
\begin{eqnarray*}
2x^3 - x^2 + x + 3 &= \alpha_1 x^3 + \alpha_1 x^2 + \alpha_1 x + \alpha_1 + \alpha_2 x^2 + \alpha_2 x + \alpha_2 +\alpha_3 x + \alpha_3 \\
 &= x^3 (\alpha_3)  + x^2 (\alpha_2 + \alpha_1) + x(\alpha_3 + \alpha_2 + \alpha_1) + \alpha_1 + \alpha_2 + \alpha_3
 \end{eqnarray*}
 \begin{align*}
	\alpha_3 &=  2\\
	\alpha_2 &= -1 - \alpha_1 \\
	\alpha_2 &= -1 - 2\\
	\alpha_2 &= -3
\end{align*}

Ahora llegamos a una contradicción, puesto que el sistema de ecuaciones anterior implica que $\alpha_1 + \alpha_2 + \alpha_3 = 3 = 1$ por lo que el conjunto S no genera $ 2x^3 - x^2 + x + 3$

\textbf{d)} $$ \begin{pmatrix} 1 & 2 \\ -3 & 4 \end{pmatrix},  S =  \lbrace \begin{pmatrix} 1 & 0 \\ -1 & 0 \end{pmatrix} , \begin{pmatrix} 0 & 1 \\ 0 &1 \end{pmatrix} , \begin{pmatrix} 1 & 1 \\ 0 &0 \end{pmatrix} \rbrace$$
Recordemos que la suma de matrices se hace entrada por entrada eso es, si se van a sumar 2 matrices $A + B$ se hace de la forma $a_{ij} + b_{ij}  \forall i,j \in A,B$ de tal manera que existen  $a_{ij} + b_{ij}  \forall i,i \in A,B$ de tal manera que existen $\alpha, \beta, \gamma \tq $

\begin{align*}
 \begin{pmatrix} 1 & 2 \\ -3 & 4 \end{pmatrix} &= \alpha \begin{pmatrix} 1 & 0 \\ -1 & 0 \end{pmatrix}  + \beta \begin{pmatrix} 0 & 1 \\ 0 &1 \end{pmatrix}  + \gamma  \begin{pmatrix} 1 & 1 \\ 0 &0 \end{pmatrix} \\
  & = \begin{pmatrix} \alpha_{1,1}  + \gamma_{1,1} & \beta_{1,2} + \gamma_{1,2} \\ -\alpha_{2,1} & \beta_{2,2} \end{pmatrix}  
\end{align*}

Notemos que $$ -\alpha_{2,1} = -3  \implies  \alpha = 3$$ y luego 
$$\beta_{2,2} = 4 \implies \beta = 4 $$ y finalmente $$\gamma = 2 - \beta_{1,2} \implies \gamma = 4 $$

\noindent \textbf{7}. Determina cuando los siguientes conjuntos son linealmente dependientes o linealmente independientes. \\
  
\textbf{a)} $\left \lbrace \begin{pmatrix} 1 & -3 \\ -2 & 4 \end{pmatrix} , \begin{pmatrix} -2 & 6 \\ 4 & -8 \end{pmatrix} \right \rbrace \in M_{2x2}(\mathbb{R})$\\

Sean $\alpha _1, \alpha _2, \in \mathbb{R}$. Entonces:\\

$\alpha_1 \begin{pmatrix} 1 & -3 \\ -2 & 4 \end{pmatrix}  + \alpha_2 \begin{pmatrix} -2 & 6 \\ 4 & -8 \end{pmatrix} = \begin{pmatrix}
0 & 0 \\ 0 & 0 \end{pmatrix}$\\

Ahora:\\

$\begin{pmatrix} \alpha_1 & -3\alpha_{1} \\ -2\alpha_{1} & 4\alpha_{1} \end{pmatrix} +
\begin{pmatrix} -2\alpha_{2} & 6\alpha_2 \\ 4\alpha_{2} & -8\alpha_{2} \end{pmatrix} = \begin{pmatrix} 	0 & 0 \\ 0 & 0 \end{pmatrix}$\\

Sumamos cada elemento de las matrices al correspondiente reglón y columna:\\

\begin{center}
$\begin{pmatrix}
\alpha_{1}-2\alpha_{2} & -3\alpha_{1}+6\alpha_{2} \\ -2\alpha_{1}+4\alpha_{2} & 4\alpha_{1}-8\alpha_{2}
\end{pmatrix} = \begin{pmatrix}
0 & 0 \\ 0 & 0
\end{pmatrix}$
\end{center}

Tenemos el siguiente sistema de ecuaciones:
\begin{center}
\begin{align*}
	\alpha_{1}-2\alpha_{2} = 0\\
	-3\alpha_{1} + 6\alpha_{2} = 0\\
	-2\alpha_{1}+4\alpha_{2}= 0\\
	4\alpha_{1}-8\alpha_{2}= 0
\end{align*}
\end{center}
Multiplicamos dos veces el renglón 3 y lo sumamos al renglón 4. También multiplicamos dos veces el renglón 1 y lo sumamos al renglón 3.
\begin{center}
\begin{align*}	
	\alpha_{1}-2\alpha_{2} = 0\\
	-3\alpha_{1}+6\alpha_{2}= 0\\
	0\alpha_{1}+0\alpha_{2}=0\\
	0\alpha_{1}+0\alpha_{2}= 0
\end{align*}
\end{center}
Por último multiplicamos tres veces el renglón 1 y lo sumamos al renglón 2:
\begin{center}
\begin{align*}
	\alpha_{1}-2\alpha_{2}=0\\
	0\alpha_{1}+0\alpha_{2}=0
\end{align*}	
\end{center}
Entonces $\alpha_{1}=2\alpha_{2}$.\\ Esto indica que $\alpha_{1}$ depende de $\alpha_{2}$. Por lo tanto, el conjunto  $\begin{pmatrix} 1 & -3 \\ -2 & 4 \end{pmatrix} , \begin{pmatrix} -2 & 6 \\ 4 & -8 \end{pmatrix}\in M_{2x2}$  es \textbf{linealmente dependiente}.

\textbf{b)} $\left \lbrace \begin{pmatrix} 1 & -2 \\ -1 & 4 \end{pmatrix} ,\begin{pmatrix} -1 & 1 \\ 2 & -4 \end{pmatrix}  \right \rbrace \in M_{2x2}(\mathbb{R})$\\

Sean $\alpha _1, \alpha _2, \in \mathbb{R}$. Entonces:\\

$\alpha_1 \begin{pmatrix} 1 & -2 \\ -1 & 4 \end{pmatrix}  + \alpha_2 \begin{pmatrix} -1 & 1 \\ 2 & -4 \end{pmatrix} = \begin{pmatrix}
0 & 0 \\ 0 & 0 \end{pmatrix}$\\

Ahora:\\

$\begin{pmatrix} \alpha_1 & -2\alpha_{1} \\ -\alpha_{1} & 4\alpha_{1} \end{pmatrix} +
\begin{pmatrix} -\alpha_{2} & \alpha_2 \\ 2\alpha_{2} & -4\alpha_{2} \end{pmatrix} = \begin{pmatrix} 	0 & 0 \\ 0 & 0 \end{pmatrix}$\\

Sumamos cada elemento de las matrices al correspondiente reglón y columna:\\

\begin{center}
	$\begin{pmatrix}
	\alpha_{1}-\alpha_{2} & -2\alpha_{1}+\alpha_{2} \\ -\alpha_{1}+2\alpha_{2} & 4\alpha_{1}-4\alpha_{2}
	\end{pmatrix} = \begin{pmatrix}
	0 & 0 \\ 0 & 0
	\end{pmatrix}$
\end{center}

Tenemos el siguiente sistema de ecuaciones:
\begin{center}
	\begin{align*}
	\alpha_{1}-\alpha_{2} = 0\\
	-2\alpha_{1} + \alpha_{2} = 0\\
	-\alpha_{1}+2\alpha_{2}= 0\\
	4\alpha_{1}-4\alpha_{2}= 0
	\end{align*}
\end{center}

Multiplicamos cuatro veces el renglón 1 y lo restamos al renglón 4. También sumamos el renglón 1 al renglón 2:
\begin{center}
	\begin{align*}
	\alpha_{1}-\alpha_{2} = 0\\
	-2\alpha_{1} + \alpha_{2} = 0\\
	0\alpha_{1}+\alpha_{2}= 0\\
	0\alpha_{1}+0\alpha_{2}= 0
	\end{align*}
\end{center}
Tenemos que $\alpha_{2}= 0$, Entonces lo sustituimos en las demás ecuaciones:
\begin{center}
	\begin{align*}
	\alpha_{1}-0 = 0\\
	-2\alpha_{1} + 0 = 0\\
	\end{align*}
\end{center}
Es claro notar que $\alpha_{1}=0$ y $\alpha_{2}=0$.\\ Cómo ambos valen 0, podemos concluir que el conjunto $\begin{pmatrix} 1 & -2 \\ -1 & 4 \end{pmatrix} ,\begin{pmatrix} -1 & 1 \\ 2 & -4 \end{pmatrix}\in M_{2x2}(\mathbb{R})$ es \textbf{linealmente independiente}.

\textbf{c)} $\lbrace x^{3} + 2x^{2}, -x^{2} + 3x + 1, x^{3} - x^{2} + 2x -1 \rbrace \in P_{3}(\mathbb{R})$ \\

Sean $\alpha_{1}, \alpha_{2}, \alpha_{3}\in \mathbb{R}$.

Tenemos que: $0x^{3}+0x^{2}+0x+d = \alpha_{1}(x^{3}+2x^{2})+ \alpha_{2}(-x^{2}+3x+1)+ \alpha_{3}(x^{3}-x^{2}+2x-1)$ \\  
        
 $x^{3}+0x^{2}+0x+0=(\alpha_{1}+\alpha_{3})x^{3}+(2\alpha_{1}-\alpha_{2}-\alpha_{3})x^{2}+(3\alpha_{2}+2\alpha_{3})x+(\alpha_{2}-\alpha_{3})$.
 
 Obtenemos el siguiente sistema de ecuaciones:
 \begin{center}
 	\begin{align*}
 	\alpha_{1}+0\alpha_{2}+\alpha_{3}=0\\
 	2\alpha_{1}-\alpha_{2}-\alpha_{3}=0\\
 	0\alpha_{1}+3\alpha_{2}+2\alpha_{3}=0\\
 	0\alpha_{1}+\alpha_{2}-\alpha_{3}=0
 	\end{align*}
 \end{center}

Ahora multiplicamos -3 veces el renglón 4 y le sumamos el renglón 1:
\begin{center}
	\begin{align*}
	\alpha_{1}+0\alpha_{2}+\alpha_{3}=0\\
	2\alpha_{1}-\alpha_{2}-\alpha_{3}=0\\
	0\alpha_{1}+3\alpha_{2}+2\alpha_{3}=0\\
	0\alpha_{1}+0\alpha_{2}+5\alpha_{3}=0
	\end{align*}
\end{center}

Podemos obtener que $\alpha_{3}=0$. Entonces sustituimos este valor en las ecuaciones.
\begin{center}
	\begin{align*}
	\alpha_{1}+0\alpha_{2}+0=0\\
	2\alpha_{1}-\alpha_{2}-0=0\\
	0\alpha_{1}+3\alpha_{2}+0=0\\
	\alpha_{3}=0
	\end{align*}
\end{center}

De lo anterior deducimos que $\alpha_{1}= 0$, por tanto:
\begin{center}
	\begin{align*}
	0-\alpha_{2}-0=0\\
	0+3\alpha_{2}+0=0\\
	\alpha_{3}=0
	\end{align*}
\end{center}

Entonces $\alpha_{1}=0$, $\alpha_{2}=0$, $\alpha_{3}=0$. Podemos que concluir que el conjunto $\lbrace x^{3} + 2x^{2}, -x^{2} + 3x + 1, x^{3} - x^{2} + 2x -1 \rbrace \in P_{3}(\mathbb{R})$ es \textbf{linealmente independiente}.\\

\textbf{d)} $\lbrace (1,-1,2), (1,-2,1), (1,1,4) \rbrace \in \mathbb{R}^{3}$\\

Sean $\alpha_{1}, \alpha_{2}, \alpha_{3}\in \mathbb{R}$. Entonces:

$\alpha_{1}(1,-1,2)+\alpha_{2}(1,-2,1)+ \alpha_{3}(1,1,4)=(0,0,0)$.
Ahora:\\

$(\alpha_{1},-\alpha_{1},2\alpha_{1})+(\alpha_{2},-2\alpha_{2},\alpha_{2})+ (\alpha_{3},\alpha_{3},4\alpha_{3})=(0,0,0)$. Ordenamos los escalares:\\

$(\alpha_{1}+\alpha_{2}+\alpha_{3}, -\alpha_{1}-2\alpha_{2}+\alpha_{3}, 2\alpha_{1}+\alpha_{2}+4\alpha_{3})= (0,0,0)$\\

Obtenemos el siguiente sistema de ecuaciones:
\begin{center}
	\begin{align*}
	\alpha_{1}+\alpha_{2}+\alpha_{3}=0\\
	-\alpha_{1}-2\alpha_{2}+\alpha_{3}=0\\
	2\alpha_{1}+\alpha_{2}+4\alpha_{3}=0
	\end{align*}
\end{center}

Multiplicamos dos veces el renglón 1 y lo sumamos a "menos" el renglón 3. También sumamos el renglón 1 al renglón 2.
\begin{center}
	\begin{align*}
	\alpha_{1}+\alpha_{2}+\alpha_{3}=0\\
	0\alpha_{1}-\alpha_{2}+2\alpha_{3}=0\\
	0\alpha_{1}+\alpha_{2}-2\alpha_{3}=0
	\end{align*}
\end{center}

Ahora al renglón 3 le sumamos el renglón 2. Y al renglón 1 le sumamos el renglón 2.
\begin{center}
	\begin{align*}
	\alpha_{1}+0\alpha_{2}+3\alpha_{3}=0\\
	0\alpha_{1}-\alpha_{2}+2\alpha_{3}=0\\
	0\alpha_{1}+0\alpha_{2}-0\alpha_{3}=0
	\end{align*}
\end{center}

Entonces nos queda el siguiente sistema.
\begin{center}
	\begin{align*}
	\alpha_{1}+3\alpha_{3}=0\\
	-\alpha_{2}+2\alpha_{3}=0
	\end{align*}
\end{center}

De esto podemos deducir que $\alpha_{1}=-3\alpha_{3}$, $\alpha_{2}= 2\alpha_{3}$ y $\alpha_{3}= \frac{\alpha_{2}}{2}$.\\

Entonces podemos concluir que el conjunto $\lbrace (1,-1,2), (1,-2,1), (1,1,4) \rbrace \in \mathbb{R}^{3}$ es \textbf{linealmente dependiente}.\\

\textbf{e)}$\lbrace (1,-1,2), (2,0,1),(-1,2,-1)\rbrace \in \mathbb{R}^{3}$\\

Sean $\alpha_{1}, \alpha_{2}, \alpha_{3}\in \mathbb{R}$.

$\alpha_{1}(1,-1,2)+\alpha_{2}(2,0,1)+ \alpha_{3}(-1,2,-1)=(0,0,0)$.
Ahora:\\

$(\alpha_{1},-\alpha_{1},2\alpha_{1})+(2\alpha_{2},0\alpha_{2},\alpha_{2})+ (-\alpha_{3},2\alpha_{3},-\alpha_{3})=(0,0,0)$. Ordenamos los escalares:\\

$(\alpha_{1}+2\alpha_{2}-\alpha_{3}, -\alpha_{1}+0\alpha_{2}+2\alpha_{3}, 2\alpha_{1}+\alpha_{2}-\alpha_{3})= (0,0,0)$\\

Obtenemos el siguiente sistema de ecuaciones:
\begin{center}
	\begin{align*}
	\alpha_{1}+2\alpha_{2}-\alpha_{3}=0\\
	-\alpha_{1}+0\alpha_{2}+2\alpha_{3}=0\\
	2\alpha_{1}+\alpha_{2}-\alpha_{3}=0
	\end{align*}
\end{center}

Primero multiplicamos dos veces el renglón 1 y lo restamos al renglón 3. Luego sumamos el renglón 2 al renglón 1.
\begin{center}
	\begin{align*}
	0\alpha_{1}+2\alpha_{2}+\alpha_{3}=0\\
	-\alpha_{1}+0\alpha_{2}+2\alpha_{3}=0\\
	0\alpha_{1}+3\alpha_{2}-\alpha_{3}=0
	\end{align*}
\end{center}

Ahora al renglón 3 le sumamos el renglón 1:

\begin{center}
	\begin{align*}
	0\alpha_{1}+2\alpha_{2}+\alpha_{3}=0\\
	-\alpha_{1}+0\alpha_{2}+2\alpha_{3}=0\\
	0\alpha_{1}+5\alpha_{2}-0\alpha_{3}=0
	\end{align*}
\end{center}

De lo anterior obtenemos que $\alpha_{2}= 0$ y sustituimos en las demás ecuaciones.

\begin{center}
	\begin{align*}
	0+\alpha_{3}=0\\
	-\alpha_{1}+2\alpha_{3}=0\\
	\alpha_{2}=0
	\end{align*}
\end{center}

Es fácilmente apreciar que $\alpha_{1}=0$, $\alpha_{2}=0$ y $\alpha_{3}=0$\\

Por lo tanto, podemos concluir que el conjunto $ (1,-1,2), (2,0,1),(-1,2,-1) \in \mathbb{R}^{3}$ es \textbf{linealmente independiente}\\

Recuerde que $P_{n}(\mathbb{R}) = \{ a_0 +  a_1x +  a_2x^2 + \cdots +  a_n x^n \big\vert a_k \in \mathbb{R} \,  \forall k = 0,1,2,\dots n\}$
\\

\noindent \textbf{8.} ¿Cuáles de los siguientes conjuntos son bases para $\mathbb{R}^{3}$?
	\begin{definition}
		Una \textbf{base} $ \beta $ de $ \mathcal{V} $ espacio vectorial es un subconjunto de $ \mathcal{V} \tq \beta$  genera a $ \mathcal{V} $ y $ \beta $ es linealmente independiente
	\end{definition}

	a)  $S = \lbrace (1,0,-1),(2,5,1),(0,-4,3) \rbrace$
	En primer lugar veamos quién es el generado del conjunto $ S $, recordemos que un conjunto genera a otro $ \forall \hat{x} \in \mathcal{V} $ es una combinación lineal de elementos de $ \mathcal{S} $ 
	
	Sean $ \alpha_1, \beta, \gamma \in \R $ entonces
	\[ \alpha(1,0,-,) + \beta(2,5,1) + \gamma(0,-4,3)  \]
	\begin{align*}
		\alpha(1,0,-1)+ \beta(2,5,1)+ \gamma(0,-4,3) & = (\alpha,0,-\alpha) + (2\beta,5\beta,\beta) + (0,-4\gamma,3\gamma)\\
		& = (\alpha + 2\beta ,5\beta -4\gamma,-\alpha + \beta +3\gamma)
	\end{align*}
	Necesitamos que cada uno de esos vectores pueda ser el valor de una posición de $ \R^3 $ por lo que debería verse como \[ 	\alpha(1,0,-1)+ \beta(2,5,1)+ \gamma(0,-4,3) = \delta(1,0,0) + \epsilon(0,1,0) + \eta(0,0,1) \] De esta manera podemos obtener el siguiente sistema de ecuaciones
	\begin{align*}
		\alpha + 2\beta + 0\gamma = \delta + 0\epsilon + 0\eta\\
		0\alpha + 5\beta -4\gamma = 0\delta + \epsilon + 0\eta\\
		-\alpha + \beta +3\gamma = 0\delta +0\epsilon + \eta
	\end{align*}
	
	\begin{align*}
		\begin{pmatrix}
			\alpha & 2\beta & 0\gamma\\
			0\alpha & 5\beta & -4\gamma\\
			-\alpha & \beta & 3\gamma
		\end{pmatrix}
		&= 		\begin{pmatrix}
		\delta & 0\epsilon & 0\eta\\
		0\delta & \epsilon & 0\eta\\
		0\delta & 0\epsilon & \eta
		\end{pmatrix}  = 		 \begin{pmatrix}[ccc|ccc]
		\alpha & 2\beta & 0\gamma & \delta & 0\epsilon & 0\eta\\
		0\alpha & 5\beta & -4\gamma& 0\delta & \epsilon & 0\eta\\
		-\alpha & \beta & 3\gamma & 0\delta & 0\epsilon & \eta
		\end{pmatrix} 
	\end{align*}
	\begin{align*}
		 \begin{pmatrix}[ccc|ccc]
		 \alpha & 2\beta & 0\gamma & \delta & 0\epsilon & 0\eta\\
		 0\alpha & 5\beta & -4\gamma& 0\delta & \epsilon & 0\eta\\
		 -\alpha & \beta & 3\gamma & 0\delta & 0\epsilon & \eta
		 \end{pmatrix} &= 		 \begin{pmatrix}[ccc|ccc]
		 \alpha & 2\beta & 0\gamma & \delta & 0\epsilon & 0\eta\\
		 0\alpha & 5\beta & -4\gamma& 0\delta & \epsilon & 0\eta\\
		 0\alpha & 3\beta & 3\gamma & \delta & 0\epsilon & \eta
		 \end{pmatrix} && \text{1ra fila + 2da fila en \textbf{3ra fila}} \\
		 \begin{pmatrix}[ccc|ccc]
		 \alpha & 2\beta & 0\gamma & \delta & 0\epsilon & 0\eta\\
		 0\alpha & 5\beta & -4\gamma& 0\delta & \epsilon & 0\eta\\
		 0\alpha & 3\beta & 3\gamma & \delta & 0\epsilon & \eta
		 \end{pmatrix} &= \begin{pmatrix}[ccc|ccc]
		 \alpha & 2\beta & 0\gamma & \delta & 0\epsilon & 0\eta\\
		 0\alpha & 1\beta & -\frac{4}{5}\gamma& 0\delta & \frac{1}{5}\epsilon & 0\eta\\
		 0\alpha & 3\beta & 3\gamma & \delta & 0\epsilon & \eta
		 \end{pmatrix} && \text{2da fila $ \* \frac{1}{5} $ en \textbf{2da fila}}\\
		 \begin{pmatrix}[ccc|ccc]
		 \alpha & 2\beta & 0\gamma & \delta & 0\epsilon & 0\eta\\
		 0\alpha & 1\beta & -\frac{4}{5}\gamma& 0\delta & \frac{1}{5}\epsilon & 0\eta\\
		 0\alpha & 3\beta & 3\gamma & \delta & 0\epsilon & \eta
		 \end{pmatrix} &= \begin{pmatrix}[ccc|ccc]
		 \alpha & 2\beta & 0\gamma & \delta & 0\epsilon & 0\eta\\
		 0\alpha & 1\beta & -\frac{4}{5}\gamma& 0\delta & \frac{1}{5}\epsilon & 0\eta\\
		 0\alpha & 0\beta & \frac{27}{5}\gamma & \delta & -\frac{3}{5}\epsilon & \eta
		 \end{pmatrix} && \text{2da fila $ \* -3  +  3^{ra}$ fila en \textbf{3ra fila}}\\
		 \begin{pmatrix}[ccc|ccc]
		 \alpha & 2\beta & 0\gamma & \delta & 0\epsilon & 0\eta\\
		 0\alpha & 1\beta & -\frac{4}{5}\gamma& 0\delta & \frac{1}{5}\epsilon & 0\eta\\
		 0\alpha & 0\beta & \frac{27}{5}\gamma & \delta & -\frac{3}{5}\epsilon & \eta
		 \end{pmatrix} & = \begin{pmatrix}[ccc|ccc]
		 \alpha & 2\beta & 0\gamma & \delta & 0\epsilon & 0\eta\\
		 0\alpha & 1\beta & -\frac{4}{5}\gamma& 0\delta & \frac{1}{5}\epsilon & 0\eta\\
		 0\alpha & 0\beta & \gamma & \frac{5}{27}\delta & -\frac{1}{9}\epsilon & \frac{5}{27}\eta
		 \end{pmatrix} && \text{ 3ra fila $ \* \frac{5}{27}$ \textbf{en 3ra fila} }\\
		 \begin{pmatrix}[ccc|ccc]
		 \alpha & 2\beta & 0\gamma & \delta & 0\epsilon & 0\eta\\
		 0\alpha & 1\beta & -\frac{4}{5}\gamma& 0\delta & \frac{1}{5}\epsilon & 0\eta\\
		 0\alpha & 0\beta & \gamma & \frac{5}{27}\delta & -\frac{1}{9}\epsilon & \frac{5}{27}\eta
		 \end{pmatrix} &= \begin{pmatrix}[ccc|ccc]
		 \alpha & 2\beta & 0\gamma & \delta & 0\epsilon & 0\eta\\
		 0\alpha & 1\beta & 0\gamma& \frac{4}{27}\delta & \frac{1}{9}\epsilon & \frac{4}{27}\eta\\
		 0\alpha & 0\beta & \gamma & \frac{5}{27}\delta & -\frac{1}{9}\epsilon & \frac{5}{27}\eta
		 \end{pmatrix} && \text{(3ra fila $ \* \dfrac{4}{5}) + 2^{da} $  fila \textbf{en 2da fila} }\\
		 \begin{pmatrix}[ccc|ccc]
		 \alpha & 2\beta & 0\gamma & \delta & 0\epsilon & 0\eta\\
		 0\alpha & 1\beta & 0\gamma& \frac{4}{27}\delta & \frac{1}{9}\epsilon & \frac{4}{27}\eta\\
		 0\alpha & 0\beta & \gamma & \frac{5}{27}\delta & -\frac{1}{9}\epsilon & \frac{5}{27}\eta
		 \end{pmatrix} & = \begin{pmatrix}[ccc|ccc]
		 \alpha & 0\beta & 0\gamma & \frac{19}{27}\delta & -\frac{2}{9}\epsilon & -\frac{8}{27}\eta\\
		 0\alpha & 1\beta & 0\gamma& \frac{4}{27}\delta & \frac{1}{9}\epsilon & \frac{4}{27}\eta\\
		 0\alpha & 0\beta & \gamma & \frac{5}{27}\delta & -\frac{1}{9}\epsilon & \frac{5}{27}\eta
		 \end{pmatrix} && \text{(2da fila $ \* -2) + 1^{a} $ fila \textbf{en 1ra fila}}\\
		 \begin{pmatrix}[ccc|ccc]
		 \alpha & 0\beta & 0\gamma & \frac{19}{27}\delta & -\frac{2}{9}\epsilon & -\frac{8}{27}\eta\\
		 0\alpha & 1\beta & 0\gamma& \frac{4}{27}\delta & \frac{1}{9}\epsilon & \frac{4}{27}\eta\\
		 0\alpha & 0\beta & \gamma & \frac{5}{27}\delta & -\frac{1}{9}\epsilon & \frac{5}{27}\eta
		 \end{pmatrix} & = \begin{pmatrix}[ccc|ccc]
		 1 & 0 & 0 & \frac{19}{27} & -\frac{2}{9} & -\frac{8}{27}\\
		 0 & 1 & 0& \frac{4}{27} & \frac{1}{9} & \frac{4}{27}\\
		 0 & 0 & 1 & \frac{5}{27} & -\frac{1}{9} & \frac{5}{27}
		 \end{pmatrix} && \text{Conservando sólo \textbf{coeficientes}}
	\end{align*}
	\begin{align*}
		\alpha &= \frac{19}{27}\delta -\frac{2}{9}\epsilon -\frac{8}{27}\eta\\
		\beta &= \frac{4}{27}\delta+ \frac{1}{9}\epsilon + \frac{4}{27}\eta\\
		\delta &= \frac{5}{27}\delta -\frac{1}{9}\epsilon+ \frac{5}{27}\eta
	\end{align*}
	\[ \therefore  \mathcal{S} \text{ genera a } \R^3  \]
	
	Ahora veamos si es linealmente independiente, lo cual ocurre si la única solución para \[ \alpha(1,0,-1) + \beta(2,5,1) + \gamma(0,-4,3) = 0  \] es que \[ \alpha = \beta = \gamma = 0 \]
	
		\begin{align*}
		\alpha + 2\beta + 0\gamma =0\\
		0\alpha + 5\beta -4\gamma = 0\\
		-\alpha + \beta +3\gamma = 0
		\end{align*}
	Resolviendo dicho sistema obtenemos que 
	
		\begin{align*}
		\begin{pmatrix}[ccc|c]
		\alpha & 2\beta & 0\gamma & 0\\
		0\alpha & 5\beta & -4\gamma& 0\\
		-\alpha & \beta & 3\gamma & 0
		\end{pmatrix} &= \begin{pmatrix}[ccc|ccc]
		\alpha & 2\beta & 0\gamma & 0\\
		0\alpha & 5\beta & -4\gamma& 0\\
		0\alpha & 3\beta & 3\gamma & 0
		\end{pmatrix} && \text{1ra fila + 2da fila en \textbf{3ra fila}} \\
		\begin{pmatrix}[ccc|ccc]
		\alpha & 2\beta & 0\gamma & 0\\
		0\alpha & 5\beta & -4\gamma& 0\\
		0\alpha & 3\beta & 3\gamma & 0
		\end{pmatrix} &= \begin{pmatrix}[ccc|ccc]
		\alpha & 2\beta & 0\gamma & 0\\
		0\alpha & 1\beta & -\frac{4}{5}\gamma& 0\\
		0\alpha & 3\beta & 3\gamma & 0
		\end{pmatrix} && \text{2da fila $ \* \frac{1}{5} $ en \textbf{2da fila}}\\
		\end{align*}
		\begin{align*}
		\begin{pmatrix}[ccc|ccc]
		\alpha & 2\beta & 0\gamma & 0\\
		0\alpha & 1\beta & -\frac{4}{5}\gamma& 0\\
		0\alpha & 3\beta & 3\gamma & 0
		\end{pmatrix} &= \begin{pmatrix}[ccc|ccc]
		\alpha & 2\beta & 0\gamma & 0\\
		0\alpha & 1\beta & -\frac{4}{5}\gamma& 0\\
		0\alpha & 0\beta & \frac{27}{5}\gamma & 0
		\end{pmatrix} && \text{2da fila $ \* -3  +  3^{ra}$ fila en \textbf{3ra fila}}\\
		\begin{pmatrix}[ccc|ccc]
		\alpha & 2\beta & 0\gamma 0\\
		0\alpha & 1\beta & -\frac{4}{5}\gamma& 0\\
		0\alpha & 0\beta & \frac{27}{5}\gamma & 0
		\end{pmatrix} & = \begin{pmatrix}[ccc|ccc]
		\alpha & 2\beta & 0\gamma & 0\\
		0\alpha & 1\beta & -\frac{4}{5}\gamma& 0\\
		0\alpha & 0\beta & \gamma & 0
		\end{pmatrix} && \text{ 3ra fila $ \* \frac{5}{27}$ \textbf{en 3ra fila} }\\
		\begin{pmatrix}[ccc|ccc]
		\alpha & 2\beta & 0\gamma & 0\\
		0\alpha & 1\beta & -\frac{4}{5}\gamma& 0\\
		0\alpha & 0\beta & \gamma &0
		\end{pmatrix} &= \begin{pmatrix}[ccc|ccc]
		\alpha & 2\beta & 0\gamma & 0\\
		0\alpha & 1\beta & 0\gamma& 0\\
		0\alpha & 0\beta & \gamma & 0
		\end{pmatrix} && \text{(3ra fila $ \* \dfrac{4}{5}) + 2^{da} $  fila \textbf{en 2da fila} }\\
		\begin{pmatrix}[ccc|ccc]
		\alpha & 2\beta & 0\gamma & 0\\
		0\alpha & 1\beta & 0\gamma& 0\\
		0\alpha & 0\beta & \gamma & 0
		\end{pmatrix} & = \begin{pmatrix}[ccc|ccc]
		\alpha & 0\beta & 0\gamma & 0\\
		0\alpha & 1\beta & 0\gamma& 0\\
		0\alpha & 0\beta & \gamma & 0
		\end{pmatrix} && \text{(2da fila $ \* -2) + 1^{a} $ fila \textbf{en 1ra fila}}\\
		\begin{pmatrix}[ccc|ccc]
		\alpha & 0\beta & 0\gamma & 0\\
		0\alpha & 1\beta & 0\gamma& 0\\
		0\alpha & 0\beta & \gamma &0
		\end{pmatrix} & = \begin{pmatrix}[ccc|ccc]
		1 & 0 & 0 &0\\
		0 & 1 & 0& 0\\
		0 & 0 & 1 & 0
		\end{pmatrix} && \text{Conservando sólo \textbf{coeficientes}}
		\end{align*}
		
		De esta manera podemos concluir que $ \mathcal{S} $ es \textbf{linealmente independiente} 
		\begin{center}
			$ \therefore \mathcal{S} $ es Base para $ \R^3 $ 
		\end{center}
	b) $\lbrace (2,-4,1),(0,3,-1),(6,0,-1) \rbrace$
	
	
	c) $\lbrace (1,2,-1),(1,0,2),(2,1,1) \rbrace$\\
\noindent \textbf{9.} Diga si los siguientes $x^3-2x^2+1,4x^2-x+3 y 3x-2$ generan a $P_{3}(\R)$

Sea $ax^3+bx^2+cx+d\in(\R)$. Tomamos $\alpha_{1},\alpha_{2},\alpha_{3}\in (\R)$. Entonces:

$ax^3+bx^2+cx+d = (\alpha_{1})x^3 + (-2\alpha_{1}+4\alpha_{2})x^2 + (\alpha_{2}+3\alpha_{3})x + (\alpha_{1}+3\alpha_{2}-2\alpha_{3})$

Tenemos el siguiente sistema de ecuaciones:
\begin{center}
	\begin{align*}
	\alpha_{1}&=a\\
	-2\alpha_{1}+4\alpha_{2}&=b\\
	-\alpha_{2}+3\alpha_{3}&=c\\
	\alpha_{1}+3\alpha_{2}-2\alpha_{3}&=d
	\end{align*}
\end{center}
Sustituimos $\alpha_{1}$ en las demás ecuaciones
\begin{center}
	\begin{align*}
	\alpha_{1}&=a\\
	-2a+4\alpha_{2}&=b\\
	-\alpha_{2}+3\alpha_{3}&=c\\
	a+3\alpha_{2}-2\alpha_{3}&=d
	\end{align*}
\end{center}
Del renglón 2 es fácil apreciar cuál es el valor de $\alpha_{2}$
\begin{center}
	\begin{align*}
	\alpha_{1}&=a\\
	\alpha_{2}&=\frac{b+2a}{4}\\
	-\alpha_{2}+3\alpha_{3}&=c\\
	\alpha_{1}+3\alpha_{2}-2\alpha_{3}&=d
	\end{align*}
\end{center}
De igual forma sustituimos $\alpha_{2}$ en las demás ecuaciones.
\begin{center}
	\begin{align*}
	\alpha_{1}&=a\\
	\alpha_{2}&=\frac{b+2a}{4}\\
	-(\frac{b+2a}{4})+3\alpha_{3}&=c\\
	a+3(\frac{b+2a}{4})-2\alpha_{3}&=d
	\end{align*}
\end{center}
Observamos que tenemos 2 ecuaciones, en las cuáles sólo hay un valor a encontrar, entonces en estas dos ecuaciones procedemos a encontrar el valor de $\alpha_{3}$. Primero comenzaremos con el renglón 3.
\begin{center}
	\begin{align*}
	\alpha_{1}&=a\\
	\alpha_{2}&=\frac{b+2a}{4}\\
	3\alpha_{3}&=\frac{4c}{4}+(\frac{b+2a}{4})\\
	a+3(\frac{b+2a}{4})-2\alpha_{3}&=d
	\end{align*}
\end{center}
Ahora:
\begin{center}
	\begin{align*}
	\alpha_{1}&=a\\
	\alpha_{2}&=\frac{b+2a}{4}\\
	\alpha_{3}&=\frac{\frac{4c+b+2a}{4}}{3}\\
	a+3(\frac{b+2a}{4})-2\alpha_{3}&=d
	\end{align*}
\end{center}
Obtenemos el primer valor de $\alpha_{3}$: 
\begin{center}
	\begin{align*}
	\alpha_{1}&=a\\
	\alpha_{2}&=\frac{b+2a}{4}\\
	\alpha_{3}&=\frac{4c+b+2a}{12}\\
	a+3(\frac{b+2a}{4})-2\alpha_{3}&=d
	\end{align*}
\end{center}
Encontraremos el valor de $\alpha_{3}$ en la ecuación cuatro.
\begin{center}
	\begin{align*}
	\alpha_{1}&=a\\
	\alpha_{2}&=\frac{b+2a}{4}\\
	\alpha_{3}&=\frac{4c+b+2a}{12}\\
	a+(\frac{3b+6a}{4})-2\alpha_{3}&=d
	\end{align*}
\end{center}
\begin{center}
	\begin{align*}
	\alpha_{1}&=a\\
	\alpha_{2}&=\frac{b+2a}{4}\\
	\alpha_{3}&=\frac{4c+b+2a}{12}\\
	2\alpha_{3}&=a+(\frac{3b+6a}{4})-d
	\end{align*}
\end{center}
\begin{center}
	\begin{align*}
	\alpha_{1}&=a\\
	\alpha_{2}&=\frac{b+2a}{4}\\
	\alpha_{3}&=\frac{4c+b+2a}{12}\\
	2\alpha_{3}&=\frac{4a}{4}+(\frac{3b+6a}{4})-\frac{4d}{4}
	\end{align*}
\end{center}
\begin{center}
	\begin{align*}
	\alpha_{1}&=a\\
	\alpha_{2}&=\frac{b+2a}{4}\\
	\alpha_{3}&=\frac{4c+b+2a}{12}\\
	\alpha_{3}&=\frac{\frac{3b+10a-4d}{4}}{2}
	\end{align*}
\end{center}
\begin{center}
	\begin{align*}
	\alpha_{1}&=a\\
	\alpha_{2}&=\frac{b+2a}{4}\\
	\alpha_{3}&=\frac{4c+b+2a}{12}\\
	\alpha_{3}&=\frac{3b+10a-4d}{8}
	\end{align*}
\end{center}

Una vez que encontramos los valores $\alpha_{1}, \alpha_{2}, \alpha_{3}$, para ver que el conjunto dado genera a cualquier polinomio de grado tres, damos algún polinomio y este tendrá que poder escribirse como combinación lineal los elementos del conjunto y los escalares.\\

Elegimos el polinomio $5x^3+2x^2x+2$.Ahora encontraremos los valores de $\alpha_{1}, \alpha_{2}, \alpha_{3}$ para poder escribirlo de la manera:
\begin{center}
	$ax^3+bx^2+cx+d = (\alpha_{1})x^3 + (-2\alpha_{1}+4\alpha_{2})x^2 + (\alpha_{2}+3\alpha_{3})x + (\alpha_{1}+3\alpha_{2}-2\alpha_{3})$
\end{center} 
Utilizando los resultados del sistema de ecuaciones tenemos que:
\begin{center}
\begin{align*}
\alpha_{1}=5\\
\alpha_{2}=3\\
\alpha_{3}=6\\
\alpha_{3}=\frac{20}{12}
\end{align*}
\end{center}

Como podemos apreciar los valores de $\alpha_{3}$ no son los mismos, y esto es debido a que originalmente teníamos un sistema de 4 ecuaciones con 3 incógnitas, entonces el sistema tiene diversas soluciones y al encontrar que los resultados de las ecuaciones de $\alpha_{3}$ no son el mismo, podemos concluir que el conjunto $x^3-2x^2+1,4x^2-x+3 y 3x-2$  \textbf{NO} generan a $P_{3}(\R)$


\end{document}


