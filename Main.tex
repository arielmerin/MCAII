\documentclass[letterpaper]{article}
\usepackage[utf8]{inputenc}
\usepackage[spanish, es-nodecimaldot]{babel}
\usepackage{amssymb, amsmath}
\usepackage{graphicx}
\usepackage{lipsum}
\usepackage{dsfont}
\usepackage[margin=2cm,
vmargin={1.5cm,1.3cm},
includefoot]{geometry}
\usepackage{setspace}
\usepackage{subcaption}
\usepackage{tocloft}
\usepackage{upgreek}
\usepackage{amsthm}
\usepackage{graphicx}
\usepackage{paralist}
\usepackage{tasks}

\newcommand{\V}{\mathds{V}}

\newcommand{\F}{\mathds{F}}

\newcommand{\tq}{ \quad \cdot  \backepsilon \cdot \quad }

\newcommand{\ld}{\lim\limits_{x \to 0^{+}}}

\newcommand{\li}{\lim\limits_{x \to 0^{-}}}

\newcommand{\la}{\lim\limits_{x \to a}}

\newcommand{\R}{\mathds{R}}

\renewcommand{\*}{\cdot}

\newtheorem{theorem}{Teorema}[section]
\theoremstyle{definition}
\newtheorem{definition}{Definición}[section]

\begin{document}
		\begin{titlepage}
		\begin{center}
			\Large{\textbf{Espacios vectoriales}} \\[0.1cm]
			\huge{Matemáticas para las ciencias aplicadas II}\\[0.2cm]
			\large{Aquino Chapa Armando Abraham y Merino Peña Kevin Ariel }
			\\
			\today
		\end{center}
		\let\newpage\relax% Avoid following page break
		\hrulefill
	\end{titlepage}

	
	\noindent 1. Escribe el vector cero en $M_{3x4}(\mathbb{R})$
		\begin{definition}[Matriz]
		Una \textbf{Matriz} es un arreglo rectangular de elementos de un campo $ \F(\R) $ de la forma
		\begin{equation*}
		A_{m,n} = 
		\begin{pmatrix}
		a_{1,1} & a_{1,2} & \cdots & a_{1,n} \\
		a_{2,1} & a_{2,2} & \cdots & a_{2,n} \\
		\vdots  & \vdots  & \ddots & \vdots  \\
		a_{m,1} & a_{m,2} & \cdots & a_{m,n} 
		\end{pmatrix}
		\end{equation*}
		A los elementos $ a_{i,j} $ con $ 1 \leq j \leq n $ y $ 1 \leq i \leq m $ se les llama entradas de la matriz, a las matrices las denotamos por $ \mathds{A} $ 				\textit{(letras mayúsculas)} y al conjunto de las matrices de mn se les denota por $ M_{m\times n}(\F) $
		
	\end{definition}
	De esta manera tenemos que el vector cero de la matriz de 3 renglones por 4 columnas es aquella cuyas entradas (todas) son 0 \textit{i. e. }
		\begin{equation*}
		A_{3x4} = 
		\begin{pmatrix}
		a_{1,1} & a_{1,2} & a_{1,3} & a_{1,4} \\
		a_{2,1} & a_{2,2} & a_{2,3} & a_{2,4} \\
		a_{3,1} & a_{3,2} & a_{3,3} & a_{3,4} 
		\end{pmatrix}
		= 
		\begin{pmatrix}
		0 & 0 & 0 & 0\\
		0 & 0 & 0 & 0 \\
		0 & 0 & 0 & 0
		\end{pmatrix}
		\end{equation*}
		
 \noindent2. Sea $V$ el conjunto de todas las funciones diferenciables definidas en $\mathbb{R}$. Muestre que $V$ es un espacio vectorial con las operaciones usuales de suma y multiplicación por un escalar para funciones. 
	
	Veamos que la derivada cumple las siguientes propiedades
	\begin{align*}
	(f(x) + g(x))' &= \lim_{h\to 0} \dfrac{f(x +h) + g(x + h) - (f(x) + g(x))}{h}\\
	& = \lim_{h\to 0} \dfrac{f(x+h) - f(x) + g(x+h) - g(x)}{h}\\
	&= \lim_{h\to 0} \dfrac{f(x+h)-f(x)}{h} + \lim_{h\to 0} \dfrac{g(x+h)-g(x)}{h}\\
	&=f'(x) + g'(x)
	\end{align*}
	Así hemos probado que la derivada abre sumas
	\begin{align*}
	(cf(x))' &= \lim_{h\to 0} \dfrac{cf(x + h) - cf(x)}{h}\\
	& = c\lim_{h\to 0} \dfrac{f(x+h)-f(x)}{h}\\
	&= cf'(x)
	\end{align*}
	De esta manera queda conolidado que en la funciónd derivada, los escalares son sacados de la función
	
	\begin{align*}
	f'(x) &= \lim_{h\to 0} \dfrac{f(x + h) - f(x)}{h}\\
	& = \lim_{h\to 0} \dfrac{c-c}{h}\\
	&= 0
	\end{align*}
	Esto se vale para cualquier constante, en particular el 0\\
	
 \noindent 3. Prueba que el conjunto de las funciones pares en $\mathbb{R}$ es un espacio vectorial con suma y multiplicación por escalar usuales para funciones. Recuerde que una función es par si $\forall x \in Dom(f)$ entonces $f(-x) = f(x)$ \\
 	
 	Si tenemos en cuenta que $f(-t) + g(-t) = f(t) + g(t)$ y que si tenemos constantes siempre ocurre que $cf(-t) = cf(t) $ entonces ya hemos probado las dos primeras condiciones y para hallar el neutro basta con usar el 0 del cambo $(\R)$ para notar que también lo manda al 0 vector.\\
 	
\noindent 4. Sea $V$ el conjunto de pares ordenados de números reales. Si $(a_{1},a_{2})$ y $(b_{1},b_{2})$ son elementos de $V$ y $\alpha \in \mathbb{R}$, definamos la suma y multiplicación escalar de la siguiente manera:
\begin{enumerate}
\item[(i)] $(a_{1},a_{2}) + (b_{1},b_{2}) = (a_{1} + b_{1}, a_{2}b_{2})$ 
\item[(ii)] $\alpha(a_{1},a_{2}) = (\alpha a_{1},a_{2})$.\\
\end{enumerate}
¿Es $V$ un espacio vectorial sobre $\mathbb{R}$ con estas operaciones?\\

No puede ser un espacio vectorial porque si tenemos que \[ 0(a_1, a_2) = (0, a_2)  \] para cumplir el cero vector, entonces se compliría para cualquier $a_2$ lo cual no es posible pues contradice la unicidad del cero.\\

\noindent 5. Determinar cuales de los siguientes conjuntos son subespacios de $\mathbb{R}^{3}$ bajo las operaciones de suma y multiplicación por un escalar usual.\\
Basta con revisar si en cada caso se encuentra cerrado bajo la multiplicación escalar y bajo la adición, además que siempre contenga al o

\begin{tasks}(1)
\task $W_{1} = \lbrace (a_{1},a_{2},a_{3}) \in \mathbb{R}^{3} \big\vert  a_{1}=3a_{2}$ y $a_{3}=-a_{2} \rbrace$

Sí es linear porque lo satisface $t(3,1,-1)$
\task $W_{2} = \lbrace (a_{1},a_{2},a_{3}) \in \mathbb{R}^{3} \big\vert  a_{1} = a_{3} + 2 \rbrace$

NO es linear porque no contiene al elemento neutro dentro del espacio vectorial.
\task $W_{3} = \lbrace (a_{1},a_{2},a_{3}) \in \mathbb{R}^{3} \big\vert  2a_{1} - 7a_{2} + a_{3} = 0 \rbrace$

Sí es linear pues lo satisface $(2,-7,1)$
\task $W_{4} = \lbrace (a_{1},a_{2},a_{3}) \in \mathbb{R}^{3} \big\vert  a_{1} - 4a_{2} - a_{3} = 0 \rbrace$

Sí es linear pues lo satisface $(1,-4,-1)$

\end{tasks}


\noindent 6. En cada caso diga si los vectores son generados por el conjunto $S$

\begin{tasks}(1)
\task $(2,-1,1), S =  \lbrace (1,0,2),(-1,1,1) \rbrace$

Sea $\alpha _1$, $\alpha _2 \in \mathbb{R}$.

Entonces $(2,-1,1) = \alpha _{1}(1,0,2) + \alpha _{2}(-1,1,1) = (\alpha _{1}, 0, 2\alpha _{1})+ (-\alpha _{2},\alpha _{2},\alpha _{2}) = \alpha _{1}-\alpha _{2},\alpha _{2},2\alpha _{1}+\alpha _{2}$.

Tenemos el siguiente sistema de ecuaciones:
\begin{eqnarray*}
\alpha_{1}-\alpha_{2}= 2 \\
\alpha_{2} = -1\\
2\alpha_{1}+\alpha_{2}=1\\
\end{eqnarray*}
Ahora:
\begin{eqnarray*}
\alpha_{1} -(-1)=2 \\
\alpha_{2}=-1\\
2\alpha_{1} + \alpha_{2} = 1\\
\end{eqnarray*}
Al resolver el sistema, obtenemos:
\begin{eqnarray*}
\alpha_{1}=1 \\
\alpha_{2} = -1 \\
1 = 1
\end{eqnarray*}
Entonces:\\ $1(1,0,2) + (-1)(-1,1,1) = (1,0,2) + (1,-1,-1) = (2,-1,1)$\\
Cómo el sistema de ecuaciones si se satisface, el conjunto $S$ SI genera al vector $(2,-1,-1)$
\task $(2,-1,1,3), S =  \lbrace (1,0,1,-1),(0,1,1,1) \rbrace$

Sea $\alpha _1$, $\alpha _2 \in \mathbb{R}$.\\
Entonces: $(2,-1,1,3)=\alpha_{1}(1,0,1,-1)+\alpha_{2}(0,1,1,1)= (\alpha_{1},0,\alpha_{1},-\alpha_{1})+(0,\alpha_{2},\alpha_{2},\alpha_{2})= \alpha_{1},\alpha_{2},\alpha_{1}+\alpha_{2},-\alpha_{1}+\alpha_{2}$.\\
Tenemos el siguiente sistema de ecuaciones:\\
\begin{eqnarray*}
\alpha_{1}=2\\
\alpha_{2} = -1\\
\alpha_{1}+\alpha_{2}=1\\
-\alpha_{1}+\alpha_{2}=3\\
\end{eqnarray*}
Ahora:\\
\begin{eqnarray*}
\alpha_{1}=2\\
\alpha_{2}= -1\\
2-1  =1\\
-(-1)+2 = 3\\
\end{eqnarray*}
Por último:
\begin{eqnarray*}
\alpha_{1}=2\\
\alpha_{2}=-1\\
1=1\\
3=3\\
\end{eqnarray*}
Al resolver el sistema de ecuaciones verificamos si el conjunto $S$ genera al vector. Entonces:\\
$2(1,0,1,-1)+(-1)(0,1,1,1)=(2,0,2,-2)+(0,-1,-1,-1)=(2,-1,-1,-3)$\\
Como el producto de los escalares por los elementos del conjunto $S$ no forman al vector, podemos concluir que $S$ NO genera a $(2,-1,1,3)$.
\task $2x^3 - x^2 + x + 3, S =  \lbrace x^3 + x^2 + x +1, x^2 + x +1, x +1 \rbrace$
\task $ \begin{pmatrix} 1 & 2 \\ -3 & 4 \end{pmatrix},  S =  \left \lbrace \begin{pmatrix} 1 & 0 \\ -1 & 0 \end{pmatrix} , \begin{pmatrix} 0 & 1 \\ 0 &1 \end{pmatrix} , \begin{pmatrix} 1 & 1 \\ 0 &0 \end{pmatrix} \right \rbrace$
\end{tasks}
	
\end{document}
